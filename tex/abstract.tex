

Korn's type inequalities, $\|\nabla \Bu -A\|\leq K_1 \|\nabla \Bu ^T + \nabla \Bu\|,$ originating from Korn's 1906 contributions, have been instrumental in analyzing boundary value problems and linear elasticity. They are pivotal in establishing the existence of energy minimizers and have been fundamental in both linear and nonlinear shell theories, where the control of the asymptotic of the Korn constant $K_1$ where shown to also be of extreme importance. However it does not stop here, recently, it has been understood that, in fact, they also play a central role in fundamental classical questions in fracture mechanics, in particular in Griffith's model.

The first part of this thesis is dedicated to the study of the buckling of cylindrical shells under axial compression. While Koiter's theoretical formula predicts a linear relationship between the buckling load $\lambda(h)$ of the shell thickness $h$ ($h>0$ is a small parameter), experimental data have consistently pointed to $\lambda(h)\sim h^{3/2}$; \textit{i.e.}, the shell buckles at much smaller loads for small thickness.  This discrepancy, largely attributed to the shell's sensitivity to imperfections, is rigorously investigated. Our findings assert that the buckling load, when subjected to axial compression, is linked to the curvature of its cross-sectional curve. Specifically, when the cross-section is a convex curve with uniformly positive curvature, then $\lambda(h)\sim h,$ and when the cross-section curve has positive curvature except at finitely many points, then $C_1h^{8/5}\leq \lambda(h)\leq C_2h^{3/2}$ for $h$ small thickness $h>0.$  This result in particular shows that the load is in fact not sensitive to symmetry breaking in the shell geometry. 

The second focus of this work revolves around the introduction of new weighted variants of the Poincaré and Korn inequalities, adapted specifically for plates. These tailored inequalities prove invaluable in scenarios that demand the use of polar coordinates or require a clear demarcation between longitudinal and transverse directions, such as in the study of junctions of massive bodies and thin rods.  Furthermore, better control around the solid's boundary paves the way for sophisticated localization techniques and problems with different boundary conditions.

Lastly, in the concluding part, we introduce a groundbreaking Korn inequality for plates for special functions of bounded deformation ($SBD$). This advancement is essential for analyzing fractures in thin domains and for deriving dimension reduction theories via Griffith’s model, similar to traditional approaches in elasticity. The work on $SBD$ functions and fracture mechanics has taken significant strides only in the past decade, particularly with the emergence of Korn-type inequalities for general domains.  Drawing from these recent advances, this thesis demonstrates how to manage the Korn constant $K_1$ as the thickness $h$ of a plate, denoted as $\Omega_h = [0,1]^2 \times [-h,h]$, approaches zero. We prove, the expected result, that the constant asymptotic in $SBD$ is the same as in the Sobolev space, \textit{i. e.}, $K_1 \sim \frac{1}{h}$. This result is a significant step towards the development of a full-fledged dimension reduction theory for fracture mechanics in thin domains.
