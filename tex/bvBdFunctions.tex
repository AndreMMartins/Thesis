\chapter{Korn inequality in plates for functions of special bounded deformation (SBD)}
\label{chap:KornSBD}
\begin{comment}
\section{Papers}
\begin{com}
    Ignore this part( 5 pages), Just papers for me to look into when writing the introduction.
\end{com}
\subsection{Fracture Mechanics}
\begin{enumerate}
       \item \sout{Important-- Francfort G.A., Marigo J.-J.: Revisiting brittle fracture as an energy minimization problem. J. Mech.
Phys. Solids 46 (1998), 1319-1342. Not useful... But need to look for more papers in Fracture Mechanics in general}
\begin{itemize}
\item Fonseca I and Francfort G A 0884 Relaxation in BV versus quasiconvexication in $W_0^p$
a model for the interaction between fracture and damage Calculus of Variations 23 396
%\item Mumford D and Shah J 0878 Optimal approximations by piecewise smooth functions and associated variational problems Comm Pure Applied Math 31 466574
\end{itemize}
\item Bourdin B., Francfort G.A., Marigo J.J.: The variational approach to fracture. J. Elasticity 91 (2008),
5-148
\bl{Not a bad book to get some intuition on grifith theory, but not the best. It has the variational problems of fracture mechanics though. However, they just use SBV or GSBV}
\item Babadjian, J.-F., Iurlano, F., Lemenant, A.: Partial regularity for the crack set minimizing the twodimensional Griffith energy. Accepted for publication. J. Eur. Math. Soc. arXiv:1905.10298
\item Babadjian, J.-F., Giacomini, A.: Existence of strong solutions for quasi-static evolution in brittle fracture.
Ann. Sc. Norm. Super. Pisa Cl. Sci. (5) 13, 925–974 (2014)
\item Chambolle, A., Crismale, V.: Existence of strong solutions to the Dirichlet problem for the Griffith energy. Calc. Var. Partial Differ. Equ. 58(136) (2019)\bl{ Somewhat useful introduction, some Pde Regularity results on the Appendix}
%\item Chambolle, A., Crismale, V.: Equilibrium configurations for nonhomogeneous linearly elastic materials with surface discontinuities. Preprint arXiv:2006.00480
\item  Conti, S., Focardi, M., Iurlano, F.: Existence of strong minimizers for the Griffith static fracture model in dimension two. Ann. Inst. H. Poincaré Anal. Non Linéaire 36, 455–474 (2019)
%\item Francfort, G. A. and A. Mielke: 2006, ‘Existence results for a class of rate-independent material models with nonconvex elastic energies’. J. Reine Angew.

\item Dal Maso, G., G. A. Francfort, and R. Toader: 2005, ‘Quasistatic crack growth in nonlinear elasticity’. Arch. Ration. Mech. An. 176(2), 165–225.
\item Dal Maso, G. and R. Toader: 2002, ‘A Model for the Quasi-Static Growth of Brittle Fractures: Existence and Approximation Results’. Arch. Ration. Mech. An. 162,
101–135.
\item  Chambolle, A.: A density result in two-dimensional linearized elasticity, and applications.
Arch. Ration. Mech. Anal. 167, 211–233 (2003)
\item Chambolle, A., Conti, S., Francfort, G. A.: Approximation of a brittle fracture energy
with a constraint of non-interpenetration. Arch. Ration. Mech. Anal. 228, 867–889 (2018)
\item Chambolle, A., Conti, S., Iurlano, F.: Approximation of functions with small jump sets and
the existence of strong minimizers of Griffith’s energy. J. Math. Pures Appl. (9) 128, 119–139
(2019) \bl{Very interesting paper, re-look in the future, part of a series of papers from Chambolle (Look into " Existence of strong Solutions to Dirichlet problem for GE) Main topics: $W^{1,p}$ approximations, study of minimizers, strong solutions, lower bound?}

\item  Friedrich, M.: A derivation of linearized Griffith energies from nonlinear models. Arch. Ration. Mech. Anal. 225, 425–467 (2017) Zbl 1367.35169 MR 3634030
\end{enumerate}

\begin{enumerate}
%\item \sout{Special Functions of Bounded Deformation, ([8] on Ambrosio paper) Can't find it anywhere (never published, but redone in the next  paper)}
%\item \sout{Bellettini, G., Coscia, A. Dal Maso, G. Compactness and lower semicontinuity properties in Bd Important, main theorem added to Approximation and Compactness section}
%\item An approximation result for special functions with bounded deformation, Antonin Chambolle
%\item G. Gargiulo and E. Zappale, A lower semicontinuity result in SBD for surface integral functionals of fracture mechanics, Asymptot. Anal., 72 (2011), pp. 231–249
%\item Chambolle, A., Crismale, V.: Compactness and lower semicontinuity in GSBD.J.Eur.Math.Soc. 23(3), 701–719 (2021) \bl{Very important paper, some good introduction to understand better the problem we are looking at in Fracture mechanics and how to get the existence of a weak solution. A small approach to strong solutions. Added some results from the paper}
\item [2] Ambrosio, L.: Existence theory for a new class of variational problems. Arch. Ration. Mech.
Anal. 111, 291–322 (1990) Zbl 0711.49064 MR 106
\end{enumerate}
\subsection{Korn Inequalities}
\begin{enumerate}
\item A PIECEWISE KORN INEQUALITY IN SBD AND APPLICATIONS TO
EMBEDDING AND DENSITY RESULTS. Manuel Friedrich
\item S. Conti, D. Faraco, and F. Maggi, A new approach to counterexamples to L
1 estimates: Korn’s inequality, geometric rigidity, and
regularity for gradients of separately convex functions, Arch. Rat. Mech.
Anal., 175 (2005), pp. 287–300.
\item A. Chambolle, S. Conti, and G. Francfort, Korn-Poincar´e inequalities for functions with a small jump set, Preprint hal-01091710v1,
(2014). (First Paper, I have to go back. ) Indiana Univ. Math. J. 65(4), 1373–1399 (2016)
\item Friedrich, M.: A Korn–Poincaré-type inequality for special functions of bounded deformation (2015). Preprint arXiv:1503.06755
\item Friedrich, M.: A Korn-type inequality in SBD for functions with small jump sets. Math. Models Methods Appl. Sci. 27, 2461–2484 (2017)
\item Friedrich, M.: A piecewise Korn inequality in SBD and applications to embedding and density
results. SIAM J. Math. Anal. 50, 3842–3918 (2018)


\item Crismale, V., Friedrich, M., Solombrino, F.: Integral representation for energies in linear elasticity with surface discontinuities. Adv. Calc. Var. arXiv:2005.06866
\item 
\end{enumerate}
\subsection{General BD, SBD, GSBD}
\begin{enumerate}

 \item \sout{Which special functions of bounded deformation
have bounded variation? Sergio Conti, 2018} Most recent paper, some results on $SBD^p$ Some good constructions of special functions. Open Question: is $SBD^p\subset SBV?$

%\item Traces of functions of bounded deformation,
Jean-François Babadjian
\item F. Iurlano, A density result for GSBD and its application to the approximation of brittle fracture energies, Calc. Var. Partial Differential
Equations, 51 (2014), pp. 315–342.
\end{enumerate}
\subsection{Main papers }
    \begin{enumerate}
    \item Francfort G.A., Marigo J.-J.: Revisiting brittle fracture as an energy
    minimization problem. J. Mech. Phys. Solids 46 (1998), 1319-1342.
    \item Bourdin B., Francfort G.A., Marigo J.J.: The variational approach to fracture. J. Elasticity 91 (2008),
     \item A.A. Griffith, VI. The phenomena of rupture and flow in solids, Philos. Trans. R. Soc. A, Math. Phys. Eng. Sci.
    221 (582–593) (1921) 163–198, https://doi.org/10.1098/rsta.1921.0006, http://rsta.royalsocietypublishing.org/content/
    221/582-593/163.full.pdf
    23(3), 701–719 (2021)
    \item Functions of Bounded Deformation roger
    \end{enumerate}
\end{comment}
\section{Introduction}


\begin{comment}
    Introduction sentences:
\begin{enumerate}

\item The first two authors, together with G. Francfort established in [12] a Poincaré–Korn’s inequality for
functions with p-integrable strain and a small jump set. Their idea is to estimate the symmetric variation of
u on many lines having different orientations and not intersecting the jump set, thus allowing to use of the
fundamental theorem of calculus along such lines
\item On the one hand, the drawback of [12] is the lack of control on the perimeter of the exceptional set, which
prevents good estimates for the strain; nevertheless, these can be recovered through suitable mollification.
\item Functions in SBDp arise naturally in the study of geometrically linear fracture models. They have a jump set of finite (n-1)-dimensional measure and, away from the jump set, a symmetrized gradient
 %e(u) = (∇u+∇uT)/2 in Lp, p ≥ 1. We


\item  The key idea of the proof is to use the fundamental theorem of calculus
along lines that do not intersect the jump set to estimate the variation of
u. This would be immediate in the BV setting, in which one fully controls
the gradient. This would be immediate in the BV setting, in which one fully controls
the gradient. In a BD setting one only obtains control of the longitudinal
component of u. Therefore one needs to consider lines with many different
orientations, making sure to choose them so that they do not intersect the
jump set,

\end{enumerate}
\end{comment}

As we seen, in section \ref{sec:fracture}, functions of Bounded Deformation arise naturally in the study of fracture mechanics, They have $\CH^{n-1}-$rectifiable a jump set, and away from the jump set the symmetrized gradient is a Radon Measure. To understand this space better in section \ref{sec:BD} we will go over some of its properties and how it relates to the well-known space of Bounded Variation Functions. Then we will review the recent Korn inequalities in bulk domains in section \ref{sec:kornBulkSBD}. Lastly, we will go over a new proof for Korn inequality in plates for the Sobolev function in section \ref{sec:kornPlateSob} and extend it to special functions of bounded deformation in section \ref{sec:kornPlateSBD}. 

\section{Bounded Variation and Bounded Deformation functions}
\label{sec:BD}

While we care mostly about Bounded Deformation functions ($BD$), we will also expose some properties of Bounded Variation functions ($BV$). Both spaces are very similar and it will be useful to understand how they are related, additionally, $BV$ functions have been studied for a longer time so their theory is very well documented, for more details the reader can look into chapter 5 of Lawrence Evans and Ronald Gariepy book \cite{evansGa} or \cite{SBV1, SBV2, SBV3}. Besides that bounded variations functions are a bit easier to understand since they can be considered real functions, while bounded deformation functions only make sense as vector-valued functions. Most of the results for bounded deformation functions can be found in \cite{Ambrosio1997, RogerPaper, RogerBook, SBD1, SBD2}, if the reader is interested.

In simple terms, BV and BD functions are functions where their gradient and symmetric gradient, respectively, exist only in the sense of distribution. So to fully understand this chapter a good understanding of Measure Theory and Functional Analysis is necessary. We will try to make it as self-contained as possible but we will assume the reader is familiar with basic concepts and results of the field, like Hausdorff measures, absolute continuity of Measures, Lebesgue Decomposition Theorem, Riesz Representation Theorem, etc. For more details on these topics see \cite{evansGa}.

\begin{comment}
    These are other results I studied a bit but I don't think that are necessary to this exposure, What do you think?\begin{enumerate}
        \item Space of Bounded Measures
        \item  Total variation measure
        \item Absolute continuity measure
        \item Lebesgue Decomposition Theorem
        \item Riesz Representation Theorem
        \item convergence/ Lower semicontinuity
        %\item Convolution of measures
        \item Haussdorff measures
        \item Lebesgue points for Randon measure
        \item Approximate Limits
        \item Measurability and approximate continuity
        \item Trace Theorems
        \item Sobolev embedding
        \item Approximate Differentiability
        \item Regularity
        \item GBD (probably important)
    \end{enumerate}
\end{comment}

\subsection{Definitions}

As mentioned before, we will start with the definition of functions of Bounded Variation (BV). There are two equivalent ways to define this space, the first one is by using the total variation of the function (\cite[Definition 5.1]{evansGa}) and the second is to use the Riesz Representation theorem, to actually obtain a distributional derivative (\cite[Theorem 5.1]{evansGa}). 
\begin{definition}[Bounded Variation Function] Let $U\subset\R^n$ and $\Bf \in L^1(U)$. Then, $\Bf$ is a function of bounded variation in $U$ if
    $$
    \sup \left\{\int_U \Bf \operatorname{div} \BGf\  d \Bx\ ,\ \BGf \in C_c^1\left(U ; \mathbb{R}^n\right),| \BGf | \leq 1\right\}<\infty .
    $$
    We write
    $B V(U)$
    to denote the space of functions of bounded variation in $U$. 
\end{definition}

\begin{definition}[BV Structure Theorem] Let $U\subset\R^n$ and $\Bf \in L^1(U)$. Then, $\Bf$ is a function of bounded variation in $U$ if there exist a signed Radon measure $\mu$ on $U$  such that , we have
    $$
    \int_U \Bf \operatorname{div} \BGf d \Bx=-\int_U  \BGf \mu,\quad \text{for all}\quad \BGf \in C_c^1\left(U ; \mathbb{R}^n\right)
    $$
    So the weak first partial derivatives of a BV function are Randon measures and we can therefore denote
    $$D \Bf:=\mu.$$
\end{definition}
    
\begin{remark}
    The notation for the derivative changes very often in the literature. Considering signed measures simplifies a lot of the notation, but in some cases, it is not done, \cite{evansGa} for example. In this book, our notation is equivalent to:
    $$|\mu|=\|D\Bf\|\qquad \mu = [D \Bf]:=\|D \Bf\|\llcorner\sigma$$
\end{remark}
For the case of Bounded Deformation functions, the first definition is more complicated, since we just have a distributional derivative of the symmetric gradient. So we will focus on the second one, which is more intuitive, \cite{Ambrosio1997}
\begin{definition}[Bounded Deformation Function]
Let $U\subset\R^n$ and $\Bf=(f_1,       \ldots,f_n) \in L^1(U;\R^n)$. Then, $\Bf$ is a function of bounded deformation in $U$ if  the symmetric part of the distributional gradiet of $\Bf$,i.e.:
    $$E_{ij}\Bf:=\frac{1}{2}(D_{i} f_j+D_jf_i) $$
    is a Randon measure with bounded total variation in $U$ for any $i,j=1,\ldots,n$.
    We write  $B D(U)$   to denote the space of functions of bounded deformation in $U$. 
    
    Additionally, this is a Banach space for the norm
    $$\|\Bf\|_{BD(U)} = \|\Bf\|_{L^1(U)}+\sum_{i,j}|E_{ij}\Bf|(U)$$
\end{definition}

However,  we can look at each direction separately to get real value functions and get a more  intuitive definition(\cite[Definition 1.1]{RogerPaper}):
\begin{definition}
    For every $\BGx=\left(\xi_1, \ldots, \xi_n\right) \in \mathbf{R}^n$, let $D_{\BGx}$ be the distributional derivative in the direction $\BGx$ defined by $D_{\BGx} \Bv=\sum_j \xi_j D_j \Bv$ and  for every function $\Bf\in L^1(U)$ define the function $f^{\BGx}: U \rightarrow \mathbf{R}$ by $f^{\BGx}(\Bx)=(\Bf(\Bx), \BGx)$. Then $\Bf \in B D(U)$ if 
    $$
    D_{\BGx} f^{\BGx}=(E \Bf \BGx , \BGx)
    $$
     is a bounded Radon measure on $U$ for every $\BGx$ of the form $\Be_i+\Be_j,\  i, j=1, \ldots, n$.
     
    Conversely, if $\Bf \in B D(U)$, then $D_{\BGx} f^{\BGx}$ is a bounded Radon measure on $U$ for every $\BGx \in \mathbf{R}^n$.
\end{definition}

\subsection{Jump Set characterization}


The main reason these functions are used in fracture mechanics is because they can admit jumps to represent cracks in the material. So in this subsection, we will try to understand how to define the jump set and its properties. This is much easier to understand for real functions, so we will focus first on functions of Bounded Variations.

To start, let's recall the definition of the approximate limit for a real-valued function $f$ at a point $\Bx$(\cite[Definition 5.8]{evansGa}):


\begin{definition}
    Let $f: \mathbb{R}^n \rightarrow \mathbb{R}$.
   \begin{enumerate}
        \item  We say $\lambda(\Bx)$ is the approximate $\lim \sup$ of $f$ as $\By \rightarrow \Bx$, written $$\operatorname{ap\ }\limsup _{\By \rightarrow \Bx} f(\By)=\lambda(\Bx),$$
   if $\lambda(\Bx)$ is the infimum of the real numbers $t$ such that
   $$
   \lim _{r \rightarrow 0} \frac{\mathcal{L}^n(B_r(\Bx) \cap\{f>t\})}{\mathcal{L}^n(B_r(\Bx))}=0 .
   $$
   \item Similarly, $\mu(\Bx)$ is the approximate $\lim \inf$ of $f$ as $\By \rightarrow x$, written 
   $$\operatorname{ap\ } \liminf _{\By \rightarrow \Bx} f(\By)=\mu(\Bx),$$
   if $\mu(\Bx)$ is the supremum of the real numbers $t$ such that
   $$
   \lim _{r \rightarrow 0} \frac{\mathcal{L}^n(B_r(\Bx) \cap\{f<t\})}{\mathcal{L}^n(B_r(\Bx))}=0 .
   $$
    \end{enumerate}
   Additionally,  we say $f: \mathbb{R}^n \rightarrow \mathbb{R}$ is approximately continuous at $\Bx \in \mathbb{R}^n$ if
   $$
   \lambda(\Bx) =\mu(\Bx) =f(\Bx).
   $$
    \end{definition}
   For real-valued functions, we can consider the jump set to be the points at which the approximate limit does not exist \cite[Definition 5.9]{evansGa}, i.e. 
   $$
   J_f:=\left\{\Bx \in \mathbb{R}^n \mid \lambda(\Bx)<\mu(\Bx)\right\},
   $$

    We can notice that all the Lebesgue points are also points of approximate continuity so, for a $\CL^n$-measurable function $f$, we have that $$\CL^n(J_f)=0.$$
    However, we will see that if we assume more regularity on the function $f$, we will get more information on the jump set.
   
    In the case of vector-valued functions we need to consider all the directions in which the jump can occur, for that let's introduce one-sided Lebesgue limits \cite{Ambrosio1997},

\begin{definition} Let $\BGv$ be a unit vector in $\mathbb{R}^n, \Bx \in \mathbb{R}^n$. We define the hyperplane
$$
H_\BGv:=\left\{\By \in \mathbb{R}^n \mid \BGv \cdot(\By-\Bx)=0\right\},
$$
and the half-spaces
$$
\begin{aligned}
&H_\BGv^{+}:=\left\{\By \in \mathbb{R}^n \mid \BGv \cdot(\By-\Bx) \geq 0\right\}, \\
&H_\BGv^{-}:=\left\{\By \in \mathbb{R}^n \mid \BGv \cdot(\By-\Bx) \leq 0\right\} .
\end{aligned}
$$
\end{definition}

\begin{definition}[One-sided Lebesgue limits] Let $\Bf:\R^n\to\R^n$. We say that $\Bf$ has one-sided Lebesgue limits $\Bf_{\BGv_\Bf}^+(\Bx)$ and $\Bf_{\BGv_\Bf}^-(\Bx)$ at $\Bx\in \R^n$, with respect to a suitable direction $\BGv_f\in S^{n-1}$, if 
$$\lim _{r \rightarrow 0^{+}} \frac{1}{r^n} \int_{B_r\left(\Bx\right)\cap H^\pm_{\BGv_f}}\left|\Bf(\By)-\Bf_{\BGv_\Bf}^{\pm}(\Bx)\right| d \By=0.$$
%In this case the rescaled functions $f_{r}(y):=f(x+r y)$ converge in $L^1\left(B_1(0) ; \mathbf{R}^m\right)$, as $r \rightarrow 0^{+}$, to the function
%$$
%f_0(y):= \begin{cases}f^{+}(x), & \text { if }\left(y, \BGv_f(x)\right)>0 \\ f^{-}(x), & \text { if }\left(y, \BGv_f(x)\right)<0 .\end{cases}
%$$
\end{definition}

The Jump set $J_\Bf$ is then defined as the non-Lebesgue points such that both one-sided Lebesgue limits exist and are different for some suitable direction $\BGv_f\in S^{n-1}$,\textit{i.e.}
$$J_\Bf:=\{\Bx\in\R^n:\ \Bf_{\BGv_\Bf}^+(\Bx),\Bf_{\BGv_\Bf}^-(\Bx)\ \text{exist and }  \Bf_{\BGv_\Bf}^+(\Bx)\neq\Bf_{\BGv_\Bf}^-(\Bx),\ \text{for some } \BGv_f\in S^{n-1}\}$$
 Again we can easily see that $\CL^n(J_\Bf)=0,$ but in fact for both, $BV$ and $BD$, functions we have that the jump set is much more regular, it is countably rectifiable(\cite[Theorem 5.17]{evansGa} and \cite[Proposition 3.5]{Ambrosio1997}). This is a very strong property but general enough to model cracks in fracture mechanics.

\begin{theorem}[$\CH^{n-1}$ a.e. rectifiable] Let $\Bf\in BV(U)$ or $\Bf\in BD(U)$, then there exist countably many $C^1$-hypersurfaces, $\left\{M_k\right\}_{k=1}^{\infty}$, such that
    $$
    \mathcal{H}^{n-1}\left(J_\Bf-\bigcup_{k=1}^{\infty} M_k\right)=0.
    $$
    Additionally for each $M_k$ we have a unique $C^1$ outer vector $\BGv^k$ such that
     $$\BGv_\Bf(\Bx)=\BGv^k (\Bx),\quad \Bx\in M_k,$$
    and  $\BGv_\Bf(\Bx)$ is  $\CH^{n-1} a.e.$ continuous  in $J_\Bf$ 
\end{theorem}

\subsection{Lebesgue decomposition of Df and Ef}

Now that we understand the irregular part of $BV$ and $BD$ functions better we can look into the Lebesgue decomposition of the measures associated with these functions. This is a very important result because it will help us understand how to characterize the energy of materials, more details in \cite{Ambrosio1997}.
\begin{definition} Let $\Bf \in B V(U)$. Then we can decompose the Radon measure $D\Bf$ into
$$D \Bf=D _{ac}\Bf+D_J\Bf +D_c\Bf$$
such that:
\begin{enumerate}
\item The approximate differential of $\Bf$ is the density $\nabla \Bf$, which is the absolute continuous with respect to $\CL^n$, i.e., $$D_{ac} \Bf=\nabla \Bf \mathscr{L}^n.$$
\item The jump part of $D \Bf$ is the restriction $D_J \Bf$ of the singular part of the measure, $D_s \Bf$, to the jump set $J_\Bf$, i.e., 
$$D_\Bf \Bf:=D_s \Bf\left\llcorner J_\Bf\right .$$
\item The Cantor part of $D \Bf$ is then the remaining part of the measure, i.e., 
$$D_c \Bf:=D_s \Bf\left\llcorner\left(U \backslash J_\Bf\right)\right .$$
\end{enumerate}
\end{definition}

This can be simplified a bit more. Since the cantor part $D_c$ vanishes for all Borel sets $B$ with $\mathscr{H}^{n-1}(B)<+\infty$, we know that the jump part is the only $n-1$ dimension part of $D\Bf$. Which, as we saw before, is a countably $\CH^{n-1}-$rectifiable set, so we can rewrite it as:
$$D_J  \Bf=\left[\left(\Bf_{\BGv_\Bf}^{+}-\Bf_{\BGv_\Bf}^{-}\right) \otimes \BGv_\Bf \right]\mathscr{H}^{n-1}\llcorner J_\Bf.$$

For $BD$ functions we have similar results and conclusions so we will just present them all together:

\begin{definition} Let $\Bf \in B D(U)$. Then we can decompose the vector value Randon measure $E\Bf$ into
$$E \Bf=E _{ac}\Bf+E_J\Bf +E_c\Bf$$
such that:
\begin{enumerate}
\item The approximate symmetric differential of $\Bf$ is the density $\CE \Bf$ of $E\Bf$ with respect to $\CL^n$, \textit{i.e.}, 
$$E_{ac} \Bf=\CE \Bf \mathscr{L}^n.$$
\item The jump part of $E \Bf$ is the restriction $E_J \Bf$ of  the singular part, $E_s \Bf$, to the jump set $J_\Bf$, i.e., $$E_J \Bf :=E_s \Bf\left\llcorner J_\Bf =\left[ \left(\Bf_{\BGv_\Bf}^{+}-\Bf_{\BGv_\Bf}^{-}\right) \odot \BGv_\Bf\right] \mathscr{H}^{n-1}\left\llcorner J_\Bf\right.\right. ,$$
where  $a \odot b=\frac{1}{2}(a \otimes b+b \otimes a)$.
\item The Cantor part of $E \Bf$ is then the remaining part of the measure, i.e., $$E_c u:=E_s \Bf\left\llcorner\left(U \backslash J_\Bf\right)\right. ,$$
and vanishes for all Borel sets $B$ with $\mathscr{H}^{n-1}(B)<+\infty$.
\end{enumerate}
\end{definition}

Working with the cantor part of the measure is in general very difficult and not very realistic, so it is common to work with functions where this part of the measure vanishes:

\begin{definition}[SBV and SBD] Let $\Bf\in BV(U)$ and $\Bg\in BD(u)$. Then if,
    $$D_c \Bf = 0, \qquad\qquad E_c \Bg =0$$
they are considered special functions of bounded variation and special functions of bounded deformation respectively. Additionally, the sets of these functions are normally noted as $SBV(U)$ and $SBD(U)$.
\end{definition}

\subsection{ Structure Theorem for BD functions}

As we have seen before, an important definition for BD functions is based on looking at the real function $f^{\BGx}(x)=(\Bf(\Bx), \BGx)$, for each direction $\BGx$. So an important question to ask is how does the jump set of $f$ relate to the jump set of $\Bf^{\BGx}$. The answer is given by the following theorem but to understand the theorem completely we need to introduce some new definitions. For any $\By,\BGx\in \R^n$ and any $B\subset \R^n$ consider the following spaces:
$$
\begin{aligned}
\pi_{\BGx} &:=\left\{\By \in \mathbf{R}^n:(\By, \BGx)=0\right\}, \text{ hyperplane orthogonal to }\BGx \\
B_\By^{\BGx} &:=\{t \in \mathbf{R}: \By+t \BGx \in B\}, \text{one dimensional representation of B in } \BGx  \text{-coordinates}\\
B^{\BGx} &:=\left\{\By \in \pi_{\BGx}: B_\By^{\BGx} \neq \emptyset\right\},\text{ projection of B in $\pi_\BGx$} \\
J_\Bf^{\BGx}&:=\left\{x \in J_\Bf:\left(u^{+}(\Bx)-u^{-}(\Bx), \BGx\right) \neq 0\right\} \text{, subset of }J_\Bf\text{ with jumps in }\BGx\text{-direction}.
\end{aligned}
$$
and the following functions:
\begin{align*}
    f_\By^{\BGx}(t)&:=f^{\BGx}(\By+t \BGx)=(\Bf(\By+t \BGx), \BGx) \quad \forall t \in U_\By^{\BGx} ,\\
    \Bf^*(\Bx)&:= \begin{cases}\lim _{r \rightarrow 0} \dashint_{B(\Bx, r)} \Bf  d \By & \text { if this limit exists } \\ 0 & \text { otherwise }\end{cases}
\end{align*}

%that satisfies $\quad \mathscr{H}^{n-1}\left(J_u \backslash J_u^{\BGx}\right)=0$ for $\mathscr{H}^{n-1}$-a.e. $\BGx \in \mathbf{S}^{n-1}$.
Then theorem 4.5 of \cite{Ambrosio1997} states that:
\begin{theorem}[Structure Theorem]
Let $\Bf \in B D(U)$ and let $\BGx \in \mathbf{R}^n$ with $\BGx \neq 0$. Then
\begin{enumerate}
\item $\left( \CE \Bf \BGx, \BGx\right)=\int_{U^{\BGx}} \nabla \Bf_\By^{\BGx} d \mathscr{H}^{n-1}(\By),\quad\left|\left(\CE \Bf \BGx, \BGx\right)\right|=\int_{U^{\BGx}}\left|\nabla \Bf_\By^{\BGx}\right| d \mathscr{H}^{n-1}(\By)$.

\item For $\mathscr{H}^{n-1}$-almost every $\By \in U^{\BGx}$, the functions $\Bf_\By^\BGx$ and $\Bf_\By^{*\BGx}$ belong to $B V\left(U_\By^{\BGx}\right)$ and coincide $\mathscr{L}^1$-almost everywhere on $U_\By^{\BGx}$, the measures $\left|D \Bf_\By^{\BGx}\right|$ and $V \Bf_\By^{*\BGx}$ coincide on $U_\By^{\BGx}$, and $$(\CE \Bf(\By+t \BGx) \BGx, \BGx)=\nabla \Bf_\By^{\BGx}(t)=\left(\Bf_\By^{*\BGx}\right)^{\prime}(t)$$ for $\mathscr{L}^1$-almost every $t \in U_\By^{\BGx}$.
\item $\left(E_J \Bf \BGx, \BGx\right)=\int_{U^{\BGx}} D_J f_\By^{\BGx} d \mathscr{H}^{n-1}(\By), \quad\left|\left(E_J \Bf \BGx, \BGx\right)\right|=\int_{U^{\BGx}}\left|D_J f_\By^{\BGx}\right| d \mathscr{H}^{n-1}(\By)$.

\item $\left(J_\Bf^{\BGx}\right)_\By^{\BGx}=J_{f_\By^{\BGx}}$ for $\mathscr{H}^{n-1}$-almost every $\By \in U^{\BGx}$
$$
\begin{aligned}
&\left(\Bf^{+}(\By+t \BGx), \BGx\right)=\left(f_\By^{\BGx}\right)^{+}(t)=\lim _{s \rightarrow t^{+}} \Bf_\By^{*\BGx}(s), \\
&\left(\Bf^{-}(\By+t \BGx), \BGx\right)=\left(f_\By^{\BGx}\right)^{-}(t)=\lim _{s \rightarrow t^{-}} \Bf_\By^{*\BGx}(s),
\end{aligned}
$$
and for every $t \in\left(J_\Bf^{\BGx}\right)_\By^{\BGx}$ where the normals to $J_\Bf$ and $J_{f_\By^{\BGx}}$ are oriented so that $\left(\BGv_\Bf, \BGx\right) \geq 0$.
\item $\left(E_c \Bf \BGx, \BGx\right)=\int_{U^{\BGx}} D_c f_\By^{\BGx} d \mathscr{H}^{n-1}(\By),\left|\left(E_c \Bf \BGx, \BGx\right)\right|=\int_{U^{\BGx}}\left|D_c f_\By^{\BGx}\right| d \mathscr{H}^{n-1}(\By)$.
\end{enumerate}
\end{theorem}
There are a lot of things to understand in the previous Theorem, however the most important for this Thesis is to understand part 4. In simpler terms, it means that when we fix a direction, and look at the function $f_\By^\BGx$ we get a function of bounded variation, and the jump set of this function is the projection of the jump set of $\Bf$ in the direction $\BGx$. This is very important because it means that we can study the jump set of $\Bf$ by looking at the jump set of $f_\By^\BGx$ for almost every $\By$ and $\BGx$. 

To conclude we will add a simple Corolary about  $SBD$ functions and a final remark about the inclusion of all the spaces introduced in this section from \cite{SBD3}.
\begin{corollary}
 Let $\Bf \in B D(U)$ and let $\BGx^1, \ldots, \BGx^n$ be a basis of $\R^n$. Then the following two conditions are equivalent:
 \begin{enumerate}
 \item $\Bf \in \operatorname{SBD}(U)$.
\item  For every $\BGx=\BGx^i+\BGx^j$ with $1 \leqq i, j \leq n$, we have $f_\By^{\BGx} \in S B V\left(U_\By^{\BGx}\right)$ for $\mathscr{H}^{n-1}$ almost every $\By \in U \BGx$.
%\item The measure $\left|E_s f\right|$ is concentrated on a Borel set $B \subset U$ which is $\sigma$-finite with respect to $\mathscr{H}^{n-1}$.
\end{enumerate}
\end{corollary}

\begin{theorem}
We have $B V\left(U ; \mathbf{R}^n\right) \cap S B D(U)=S B V\left(U ; \mathbf{R}^n\right)$. Moreover the inclusions $S B V\left(U; \mathbf{R}^n\right) \subseteq S B D(U) \subseteq B D(U)$ are strict.
\end{theorem}
