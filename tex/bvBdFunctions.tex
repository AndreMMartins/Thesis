\chapter{Bounded Variation and Boundary Functions}

\section{Introduction}

For now all the bounded Variation results will come from Evans Gariepy book \cite{evansGa} and the results for Bounded Deformation functions will come mostly from Roger Temam Paper and Book \cite{RogerPaper,RogerBook} and some from Ambrosio's paper \cite{Ambrosio1997}
\begin{question}
Questions for next week:
\begin{enumerate}
\item relaxation problems? What  does it really means
\item integral representation
\item  Gama convergence
\item Should I focus on 2d results also? it might be more adequate for shells right?
\item what about go more deep in sbv? even if there are sbd results already.
\item Strong solutions would make a difference for us? It looks like they focus on linear elasticity results $$(u, \Gamma) \mapsto \int_{\Omega^{\prime} \backslash(\Gamma \cup K)} \mathbb{C} e(u): e(u) \mathrm{d} x+2 \beta \mathcal{H}^{n-1}\left((\Gamma \backslash K) \cap \Omega^{\prime}\right)$$
\item Mumford-Shah functional? is just a different approach..
\item what's the antiplane case?
\end{enumerate}
\end{question}

\begin{com}
Important things to add
\begin{enumerate}
    \item measure inequalities ( ex remark 3.3 ambrosio paper)
    \item Add Korn inequality section (maybe not)
    \item Approximation by $C^1$ functions? possible for BV, what about BD? ( this should come from Approximate Differentiability)
    \item Still not fully understand why we need GSBD, U can't prove existence of minimzers without prior bounds, you can't truncate SBD functions...
\end{enumerate}
\end{com}
\section{Papers}
\subsection{Fracture Mechanics}
\begin{enumerate}
       \item \sout{Important-- Francfort G.A., Marigo J.-J.: Revisiting brittle fracture as an energy minimization problem. J. Mech.
Phys. Solids 46 (1998), 1319-1342. Not really usefull... BUt need to look for more papers in Fracture Mechanics in general}
\begin{itemize}
\item Fonseca I and Francfort G A 0884 Relaxation in BV versus quasiconvexication in $W_0^p$
a model for the interaction between fracture and damage Calculus ofVariations 23 396
\item Mumford D and Shah J 0878 Optimal approximations by piecewise smooth functions and associated variational problems Comm Pure Applied Math 31 466574
\end{itemize}
\item Bourdin B., Francfort G.A., Marigo J.J.: The variational approach to fracture. J. Elasticity 91 (2008),
5-148
\bl{Not a bad book to get some intuition on grifith theory, but not really the best. It has the variational problems of fracture mechanics though. However they just use SBV or GSBV}
\item Babadjian, J.-F., Iurlano, F., Lemenant, A.: Partial regularity for the crack set minimizing the twodimensional Griffith energy. Accepted for publication. J. Eur. Math. Soc. arXiv:1905.10298
\item Babadjian, J.-F., Giacomini, A.: Existence of strong solutions for quasi-static evolution in brittle fracture.
Ann. Sc. Norm. Super. Pisa Cl. Sci. (5) 13, 925–974 (2014)
\item Chambolle, A., Crismale, V.: Existence of strong solutions to the Dirichlet problem for the Griffith energy. Calc. Var. Partial Differ. Equ. 58(136) (2019)\bl{ Somewhat useful introduction, some Pde Regularity results on the Appendix}
\item Chambolle, A., Crismale,V.: Equilibrium configurations for nonhomogeneous linearly elastic materials with surface discontinuities. Preprint arXiv:2006.00480
\item  Conti, S., Focardi, M., Iurlano, F.: Existence of strong minimizers for the Griffith static fracture model in dimension two. Ann. Inst. H. Poincaré Anal. Non Linéaire 36, 455–474 (2019)

\item Francfort, G. A. and A. Mielke: 2006, ‘Existence results for a class of rateindependent material models with nonconvex elastic energies’. J. Reine Angew.
Math. 595, 55–91.
\item Dal Maso, G., G. A. Francfort, and R. Toader: 2005, ‘Quasistatic crack growth in nonlinear elasticity’. Arch. Ration. Mech. An. 176(2), 165–225.
\item Dal Maso, G. and R. Toader: 2002, ‘A Model for the Quasi-Static Growth of Brittle Fractures: Existence and Approximation Results’. Arch. Ration. Mech. An. 162,
101–135.
\item  Chambolle, A.: A density result in two-dimensional linearized elasticity, and applications.
Arch. Ration. Mech. Anal. 167, 211–233 (2003)
\item Chambolle, A., Conti, S., Francfort, G. A.: Approximation of a brittle fracture energy
with a constraint of non-interpenetration. Arch. Ration. Mech. Anal. 228, 867–889 (2018)
\item Chambolle, A., Conti, S., Iurlano, F.: Approximation of functions with small jump sets and
existence of strong minimizers of Griffith’s energy. J. Math. Pures Appl. (9) 128, 119–139
(2019) \bl{Very interesting paper, re look in the future, part of a series of papers from chambolle (Look into " Existence of strong Solutions to Dirichle problem for GE) Main topics: $W^{1,p}$ approximations,study of minimizers, srong solutions, lower bound?}
\item  Dal Maso, G., Lazzaroni, G.: Quasistatic crack growth in finite elasticity with noninterpenetration. Ann. Inst. H. Poincaré Anal. Non Linéaire 27, 257–290 (2010)
\item  Friedrich, M.: A derivation of linearized Griffith energies from nonlinear models. Arch. Ration. Mech. Anal. 225, 425–467 (2017) Zbl 1367.35169 MR 3634030
\item Conti, S., Focardi, M., Iurlano, F.: A note on the Hausdorff dimension of the singular set of solutions to elasticity type systems. To appear on Commun. Contemp. Math
\end{enumerate}
\subsection{Approximation and Compactness results}
\begin{enumerate}
\item \sout{Special Funtions of Bounded Deformation, ([8] on Ambrosio paper) Can't find it anywhere (never published, but redone in the next  paper)}
\item \sout{Bellettini, G., Coscia, A. Dal Maso, G. Compactness and lower semicontinuity properties in Bd Important, main theorem added to Approximation and Compactness section}
\item An approximation result for special functions with bounded deformation, Antonin Chambolle
\item G. Gargiulo and E. Zappale, A lower semicontinuity result in SBD for surface integral functionals of fracture mechanics, Asymptot. Anal., 72 (2011), pp. 231–249
\item Chambolle, A., Crismale, V.: Compactness and lower semicontinuity in GSBD.J.Eur.Math.Soc. 23(3), 701–719 (2021) \bl{Very importat paper, some good introduction to  understand better the problem we are looking at in Fracture mechanics and how to get existence of weak solution. A small approach to strong solutions. Added some results from the paper}
\item Friedrich, M., Solombrino, F.: Quasistatic crack growth in 2d-linearized elasticity. Ann. Inst.
H. Poincaré Anal. Non Linéaire 35, 27–64 (2018) Zbl 1386.74124 MR 3739927
\item [2] Ambrosio, L.: Existence theory for a new class of variational problems. Arch. Ration. Mech.
Anal. 111, 291–322 (1990) Zbl 0711.49064 MR 106
\item Maddalena, F., Solimini, S.: Lower semicontinuity properties of functionals with free discontinuities. Arch. Ration. Mech. Anal. 159, 273–294 (2001)
\item Friedrich, M.: A compactness result in GSBVp and applications to 0-convergence for free
discontinuity problems. Calc. Var. Partial Differential Equations 58, art. 86, 31 pp. (2019)
Zbl 1418.49043 MR 3969050
\item  S. Conti, M. Focardi, F. Iurlano, Approximation of fracture energies with p-growth via piecewise affine finite elements,
ESAIM Control Optim. Calc. Var. (2019), https://doi.org/10.1051/cocv/2018021, in press, preprint, arXiv:1706.01735
\end{enumerate}
\subsection{Korn Inequalities}
\begin{enumerate}
\item A PIECEWISE KORN INEQUALITY IN SBD AND APPLICATIONS TO
EMBEDDING AND DENSITY RESULTS. Manuel Friedrich
\item S. Conti, D. Faraco, and F. Maggi, A new approach to counterexamples to L
1
estimates: Korn’s inequality, geometric rigidity, and
regularity for gradients of separately convex functions, Arch. Rat. Mech.
Anal., 175 (2005), pp. 287–300.
\item A. Chambolle, S. Conti, and G. Francfort, Korn-Poincar´e inequalities for functions with a small jump set, Preprint hal-01091710v1,
(2014). (First Paper, I have to go back. ) Indiana Univ. Math. J. 65(4), 1373–1399 (2016)

\item   korn and poincare/ korn inequalities for functions with a small jump set
\item Friedrich, M.: A Korn–Poincaré-type inequality for special functions of bounded deformation (2015). Preprint arXiv:1503.06755
\item Friedrich, M.: A Korn-type inequality in SBD for functions with small jump sets. Math. Models Methods Appl. Sci. 27, 2461–2484 (2017)
\item Friedrich, M.: A piecewise Korn inequality in SBD and applications to embedding and density
results. SIAM J. Math. Anal. 50, 3842–3918 (2018)
\end{enumerate}
\subsection{ Integral Representation}
\begin{enumerate}
\item F. Rindler, Lower semicontinuity for integral functionals in the space
of functions of bounded deformation via rigidity and Young measures,
Arch. Ration. Mech. Anal., 202 (2011), pp. 63–113.
\item F. Ebobisse and R. Toader, A note on the integral representation
representation of functionals in the space SBD(Ω), Rend. Mat. Appl.
(7), 23 (2003), pp. 189–201 (2004).
\item Conti, S., Focardi,M., Iurlano,F.: Integral representation for functionals defined on SBDp in dimension 2. Arch. Rat. Mech. Anal. 223, 1337–1374 (2017)
\item Crismale, V., Friedrich, M., Solombrino, F.: Integral representation for energies in linear elasticity with surface discontinuities. Adv. Calc. Var. arXiv:2005.06866
\end{enumerate}
\subsection{General BD, SBD, GSBD}
\begin{enumerate}

\item GENERALISED FUNCTIONS OF BOUNDED DEFORMATION
GIANNI DAL MASO \bl{(done main results in the end), very important paper to understand why shoud we use GSBD)}
\item Fine properties of functions of bounded deformation – an approach via
linear PDE,
Guido De Philippis1 and Filip Rindler2
 \item \sout{Which special functions of bounded deformation
have bounded variation? sergio conti, 2018} Most recent paper, some results on $SBD^p$ Some good constructions of especial functions. Open Question: is $SBD^p\subset SBV?$

\item Percivale D., Tomarelli F.: From SBD to SBH : the elastic-plastic plate. Interfaces Free Bound. 4
(2002), 137-165

\item Traces of functions of bounded deformation,
Jean-François Babadjian
 \item F. Iurlano, A density result for GSBD and its application to the approximation of brittle fracture energies, Calc. Var. Partial Differential
Equations, 51 (2014), pp. 315–342.
\item Sobolev inequalities for the symmetric gradient in arbitrary domains

\bl{This paper talks about Sobolev  type inequalities of irregular domains where we do not have Korn inequalities
$$\|\mathbf{u}\|_{L^{\frac{\alpha p}{n-p}}(\Omega, \mu)} \leq C\left(\|\mathcal{E} \mathbf{u}\|_{L^p(\Omega)}+\|\mathbf{u}\|_{L^{\frac{p(n-1)}{n-p}}(\partial \Omega)}\right)$$}

\end{enumerate}
\subsection{ General BV, SBV, GSBV}
\begin{enumerate}
\item Special functions with bounded variation and
with weakly differentiable traces on the jump set
Luigi Ambrosio (probably not necessary)

\item De Giorgi, E., Carriero, M., Leaci, A.: Existence theorem for a minimum problem with free discontinuity set. Arch. Rat. Mech. Anal. 108, 195–218 (1989) \bl{ Shows up a lot, some Poincare inequalities for SBV, with exact representation of the linear operator. NOT USEFULL}
\end{enumerate}

\section{Important Measure Results and Notation}

\begin{definition}[Space of Bounded Measures] Let $U$ be an open set in $\R^n$, and $\mu\in D'(U)$ be a linear transformation on $C^\infty_0(U)$ then we say that $\mu$ is  bounded if
$$
\sup_{\substack{\phi \in {C}_0^{\infty}(U) \\ |\phi(x)| \leq 1}} <\mu, \phi> < \infty$$

In this work we will call the set of all bounded measures in $U$ to be $M_1(U)$ with norm $\|\mu\|_{M_1(U)}=\int_U|\mu|=|\mu|(U)$ defined by the previous equation.
\end{definition}
\begin{remark}
Is important to notice that we will $\mu$ is not necessary a positive measure, like is use in \cite{evansGa}
\end{remark}

\begin{definition}[Total variation measure]
Let $\mu\in M_1(U)$  and let then we can define two positive measures defined on every open set $V\subset U$ (Will be more rigorous in the future)
\begin{align*}
   \mu^+&=\sup\left\{\mu(A)\mid B\subset V \right\}\\
   \mu^+&=-\inf\left\{\mu(A)\mid B\subset V \right\}
\end{align*}
so we have that $\mu=\mu^+-\mu^-$ and the total variation of $\mu$ will be defined as
$$|\mu|(V)=\mu^+(V)-\mu^-(V)=\sup_{\substack{\phi \in {C}_0^{\infty}(V) \\ |\phi(x)| \leq 1}} <\mu, \phi>.$$
Notice, $|\mu|$ is a positive Radon measure.
\end{definition}


\begin{question}
Is this rigorous enough? Or is it not correct? I will try to change all the results on Evan to get according to this so we can have a base frame work from now on
\end{question}

Additionally we can have matrix valued matrix, $\mu= \{\mu_{ij}\}$. We say that $\mu$ is a Radon/Bounded measure if $\mu_{ij}$ are Radon/bounded measures.  Additionally we can define the real-valued Randon measure
$$
(\mu \xi, \xi):=\sum_{i, j=1}^n \xi^i \xi^j \mu_{i j}, \qquad (\mu \xi, \xi)(B)= (\mu(B) \xi, \xi)
$$
It is easy to check that for any basis $\xi_1, \ldots, \xi_n$ of $\mathbf{R}^n$ there exists a constant $c$, depending on the basis, such that
$$
|\mu| \leqq c \sum_{i, j=1}^n\left|\left(\mu\left(\xi_i+\xi_j\right), \xi_i+\xi_j\right)\right|.
$$
\begin{lemma}
Let $f \in L^1\left(\mathbf{R}^n ; \mathbf{R}^n\right)$, let $\mu=\left\{\mu_{i j}\right\}$ be a symmetric matrix of Radon measures in $\mathbf{R}^n$ and assume that, for any $\xi \in \mathbf{R}^n,|(\mu \xi, \xi)|$-almost every point of $\mathbf{R}^n$ is a Lebesgue point of $f^{\xi}$. Then $|\mu|$-almost every point of $\mathbf{R}^n$ is a Lebesgue point for $f$.
\end{lemma}

\begin{question}
 All the following results are valid for signed measures right?
\end{question}
\begin{definition} Assume $\mu$ and $\nu$ are Borel measures on $\mathbb{R}^n$.
\begin{enumerate}
    \item The measure $\nu$ is absolutely continuous with respect to $\mu$, written
    $$\nu<<\mu$$
provided $\mu(A)=0$ implies $\nu(A)=0$ for all $A \subseteq \mathbb{R}^n$.
\item The measures $\nu$ and $\mu$ are mutually singular, written
$$\nu \perp \mu, \quad$$ 
if there exists a Borel subset $B \subseteq \mathbb{R}^n$ such that 
$$
\mu\left(\mathbb{R}^n-B\right)=\nu(B)=0
$$
\end{enumerate}
\end{definition}

\begin{theorem}[Lebesgue Decomposition Theorem] Let $\nu$ and $\mu$ be Radon measures on $\mathbb{R}^n$.
\begin{enumerate}
    \item Then
$$
\nu=\nu_{\mathrm{ac}}+\nu_{\mathrm{s}},
$$
where $\nu_{\mathrm{ac}}, \nu_{\mathrm{s}}$ are Radon measures on $\mathbb{R}^n$ with
$$
\nu_{\mathrm{ac}}<<\mu,\ \  \nu_{\mathrm{s}} \perp \mu \text {. }
$$
\item Furthermore,
$$
D_\mu \nu=D_\mu \nu_{\mathrm{ac}},\ \  D_\mu \nu_{\mathrm{s}}=0 \quad \mu \text {-a.e.; }
$$
and consequently
$$
\nu(A)=\int_A D_\mu \nu d \mu+\nu_s(A)
$$
for each Borel set $A \subseteq \mathbb{R}^n$.
\end{enumerate}
\end{theorem}

\begin{definition}
We call $\nu_{\mathrm{ac}}$ the absolutely continuous part and $\nu_{\mathrm{s}}$ the singular part of $\nu$ with respect to $\mu$. And this results are mainly applied for $\mu=\CL ^n$
\end{definition} 

\begin{theorem}[Riesz Representation Theorem] Let
$$
L: C_c\left(\mathbb{R}^n ; \mathbb{R}^m\right) \rightarrow \mathbb{R}
$$
be a linear functional satisfying
$$
\sup \left\{L(f)\left|f \in C_c\left(\mathbb{R}^n ; \mathbb{R}^m\right),\right| f \mid \leq 1, \operatorname{spt}(f) \subseteq K\right\}<\infty \quad(\star)
$$
for each compact set $K \subset \mathbb{R}^n$. Then there exists signed Radon measure $\mu$ on $\mathbb{R}^n$ such that
$$L(f) =\int_{\R^n} f \mu$$

for all $f \in C_c\left(\mathbb{R}^n ; \mathbb{R}^m\right)$.
Additionally we have that
$$
|\mu|(V):=\sup \left\{L(f)\left|f \in C_c\left(\mathbb{R}^n ; \mathbb{R}^m\right),\right| f \mid \leq 1, \operatorname{spt}(f) \subseteq V\right\} .
$$
\end{theorem}

\begin{question} Theorem in Evans for positive measures:
\begin{theorem}[Weak convergence of measures]. Let $\mu, \mu_k(k=1,2, \ldots)$ be Radon measures on $\mathbb{R}^n$. The following three statements are equivalent:
\begin{enumerate}
    \item $\lim _{k \rightarrow \infty} \int_{\mathbb{R}^n} f d \mu_k=\int_{\mathbb{R}^n} f d \mu$ for all $f \in C_c\left(\mathbb{R}^n\right)$
    \item(Lower semicontinuity) $\limsup _{k \rightarrow \infty} \mu_k(K) \leq \mu(K)$ for each compact set $K \subseteq \mathbb{R}^n$ and $$\mu(U) \leq \lim \inf _{k \rightarrow \infty} \mu_k(U)$$ for each open set $U \subseteq \mathbb{R}^n$.
    \item $\lim _{k \rightarrow \infty} \mu_k(B)=\mu(B)$ for each bounded Borel set $B \subseteq \mathbb{R}^n$ with $\mu(\partial B)=0$
\end{enumerate}
\end{theorem}
\end{question}

\begin{theorem}[Lower semicontinuity]
Let $\mu_k\in M_1(U)$ a sequence of measures that converge weakly to $\mu$ and $V\subset U$ then
$$
|\mu|(V) \leq \liminf _{k \rightarrow \infty}|\mu_k|(V)
$$
\end{theorem}
\begin{lemma}
Let $\mu_k\in M_1(U)$ a sequence of measures that converge weakly to $\mu$ and $|\mu_j|(U)\to|\mu|(U)$ then for any Borel set $V\subset U$, such that $|\mu|(U\cap\partial V)=0$ then
$$|\mu_k|(V)\to|\mu|(V)$$
\end{lemma}
\begin{comment}
So  basically we have i) from theorem 2.8 implies ii) and iii) dor signed measures also. the equivalence is also true  right?
\end{comment}
\begin{theorem}[Convolution with measure] Let 
function $\eta$ belonging to $\mathscr{C}_0^{\infty}\left(\mathbb{R}^n\right)$, and $\mu\in M_1(\R^n)$ then the convolution $\eta * \mu$ is a function in $\mathscr{C}^{\infty}\left(\mathbb{R}^n\right)$. Additionally we have that
\begin{enumerate}[label=(\roman*)]
\item $\int_{\mathbb{R}^n} |\eta * \mu| d x \leq \int_{\mathbb{R}^n}|\eta| d x  \int_{\mathbb{R}^n}|\mu|$
\item $\int_{\mathbb{R}^n}\left|\eta_\epsilon \star \mu\right| d x \leq \int_{\mathbb{R}^n} |\mu|$
\item
$$
\left\{\begin{array}{l}
\eta_\epsilon * \mu\rightharpoonup \mu \text { in } M_1\left(\mathbb{R}^n\right) \\
\int_{\mathbb{R}^n}\left|\eta_\epsilon * \mu\right| d x\to \int_{\mathbb{R}^n}|\mu|
\end{array}\right.
$$
\end{enumerate}

If $\mu$ is not define in $\R^n$ we can consider any compact supported function $\phi \in C^\infty_0(U)$ and consider the  measure $\phi\mu\in M_1(U)$ instead.
\end{theorem}

\begin{com}
Maybe add something about Hausdorff measures
\end{com}

For any $y,\xi\in \R^n$ and any $B\subset \R^n$ consider the following spaces:
$$
\begin{aligned}
\pi_{\xi} &:=\left\{y \in \mathbf{R}^n:(y, \xi)=0\right\}, \text{ hyperplane orthogonal to }\xi \\
B_y^{\xi} &:=\{t \in \mathbf{R}: y+t \xi \in B\}, \text{"one dimensional representation of B in } \xi  \text{-coordinates"}\\
B^{\xi} &:=\left\{y \in \pi_{\xi}: B_y^{\xi} \neq \emptyset\right\},\text{ projection of B in $\pi_\xi$} .
\end{aligned}
$$
and the following functions:
$$f^{\xi}=f^{\xi}(f, \xi)$$
$$f_y^{\xi}(t)=f^{\xi}(y+t \xi)=(f(y+t \xi), \xi) \quad \forall t \in U_y^{\xi} .$$
So now for every $y\in U^\xi$ we can consider an oscillation measure( similar to section 1.9.4 in \cite{evansGa})
$v_y\in M_1(U^\xi_y)$ such that
$$
\int_{U^{\xi}}\left|v_y\right|\left(U_y^{\xi}\right) d \CH^{n-1}(y)< \infty
$$

This hypotheses make it possible to define a bounded measure in $U$ as $\lambda = \int_{U^\xi}v_yd\CH^{n-1}(y)\in M_1(U)$ such that
$$
\lambda(B):=\int_{U^{\xi}} v_y\left(B_y^{\xi}\right) d \mathscr{H}^{n-1}(y) \quad \forall B \in \CB(U)
$$
and 
$$
|\lambda| = \int_{U^{\xi}}\left|v_y\right| d \CH^{n-1}(y)
$$

\subsection{Lebesgue points, Approximate Continuity, jump points}

\begin{theorem}[Lebesgue points for Randon measure]
Let $\mu$ be a Radon measure on $\mathbb{R}^n$ and $f \in L_{\text {loc }}^1\left(\mathbb{R}^n, \mu\right)$. Then
$$
\lim _{r \rightarrow 0} \dashint_{B(x, r)} f d \mu=f(x)
$$
for $\mu$-a.e. $x \in \mathbb{R}^n$.
Additionally, if $f \in L_{\text {loc }}^p\left(\mathbb{R}^n, \mu\right)$ for some $1 \leq p<\infty$ then
$$
\lim _{r \rightarrow 0} \dashint_{B(x, r)}|f-f(x)|^p d \mu=0
$$
for $\mu$-a.e. point $x$.
\end{theorem}

\begin{definition}
 A point $x$ for which $(\star)$ holds is called a Lebesgue point of $f$ with respect to $\mu$. When $\mu=\CL^n$ we will define
 \begin{enumerate}
     \item The Lebesgue points of $f$ in $U$ as $U_f$.
     \item The Lebesgue discontinuity set $S_f:=U-U_f$
     \item The  precise representative of $f$
$$
f^*(x):= \begin{cases}\lim _{r \rightarrow 0} \dashint_{B(x, r)} f  d y & \text { if this limit exists } \\ 0 & \text { otherwise }\end{cases}
$$
\end{enumerate}
\end{definition}



\begin{definition}
 Let $f: \mathbb{R}^n \rightarrow \mathbb{R}$.
\begin{enumerate}
     \item  We say $\lambda(x)$ is the approximate $\lim \sup$ of $f$ as $y \rightarrow x$, written $$\operatorname{ap\ }\limsup _{y \rightarrow x} f(y)=\lambda(x),$$
if $l$ is the infimum of the real numbers $t$ such that
$$
\lim _{r \rightarrow 0} \frac{\mathcal{L}^n(B(x, r) \cap\{f>t\})}{\mathcal{L}^n(B(x, r))}=0 .
$$
\item Similarly, $\mu(x)$ is the approximate $\lim \inf$ of $f$ as $y \rightarrow x$, written 
$$\operatorname{ap\ } \liminf _{y \rightarrow x} f(y)=\mu(x),$$
if $\mu(x)$ is the supremum of the real numbers $t$ such that
$$
\lim _{r \rightarrow 0} \frac{\mathcal{L}^n(B(x, r) \cap\{f<t\})}{\mathcal{L}^n(B(x, r))}=0 .
$$
 \end{enumerate}
Additionally,  we say $f: \mathbb{R}^n \rightarrow \mathbb{R}^m$ is approximately continuous at $x \in \mathbb{R}^n$ if
$$
\lambda(x) =\mu(x) =f(x).
$$
\end{definition}
Testing  
Better definition in "GENERALISED FUNCTIONS OF BOUNDED DEFORMATION
GIANNI DAL MASO"
\begin{theorem}[Measurability and approximate continuity] Suppose that $f: \mathbb{R}^n \rightarrow \mathbb{R}^m$ is $\mathcal{L}^n$-measurable.
Then $f$ is approximately continuous $\mathcal{L}^n-a . e$.
\end{theorem}
For real valued functions we can consider the jump set to be 
$$
J_f:=\left\{x \in \mathbb{R}^n \mid \lambda(x)<\mu(x)\right\},
$$
denote the set of points at which the approximate limit does not exist.


However, this definition will not work for vector-valued functions. We will have to introduce one side Lebesgue limits before.

\begin{definition} Let $\nu$ be a unit vector in $\mathbb{R}^n, x \in \mathbb{R}^n$. We define the hyperplane
$$
H_\nu:=\left\{y \in \mathbb{R}^n \mid \nu \cdot(y-x)=0\right\}
$$
and the half-spaces
$$
\begin{aligned}
&H_\nu^{+}:=\left\{y \in \mathbb{R}^n \mid \nu \cdot(y-x) \geq 0\right\}, \\
&H_\nu^{-}:=\left\{y \in \mathbb{R}^n \mid \nu \cdot(y-x) \leq 0\right\} .
\end{aligned}
$$
\end{definition}

\begin{definition}[One-sided Lebesgue limits] We say that $f$ has one-sided Lebesgue limits $f^+(x)$ and $f^-(x)$ at $x\in U$, with respect to a suitable direction $\nu_f\in S^{n-1}$, if 
$$\lim _{r \rightarrow 0^{+}} \frac{1}{r^n} \int_{B\left(x, r\right)\cap H^\pm_{\nu_f}}\left|f(y)-f^{\pm}(x)\right| d y=0.$$
In this case the rescaled functions $f_{r}(y):=f(x+r y)$ converge in $L^1\left(B_1(0) ; \mathbf{R}^m\right)$, as $r \rightarrow 0^{+}$, to the function
$$
f_0(y):= \begin{cases}f^{+}(x), & \text { if }\left(y, \nu_f(x)\right)>0 \\ f^{-}(x), & \text { if }\left(y, \nu_f(x)\right)<0 .\end{cases}
$$

The Jump set $J_f$ is then defined as the points in $S_f$ such that both one sided Lebesgue limits exist and are different.

\end{definition}


\section{Bounded Variation and Bounded Deformation Functions}

\begin{definition}[Bounded Variation Function]
A function $f \in L^1(U)$ has bounded variation in $U$ if
$$
\sup \left\{\int_U f \operatorname{div} \phi d x\left|\phi \in C_c^1\left(U ; \mathbb{R}^n\right),\right| \phi \mid \leq 1\right\}<\infty .
$$
We write
$B V(U)$
to denote the space of functions of bounded variation in $U$. 
\end{definition}

\begin{question}
Is it important to define locally bounded Variation?
\end{question}

Using Riesz representation Theorem  we can easily get the Structure Theorem for $BV_{loc}$ that I will for now take it as an equivalent definition to match BD theory.

\begin{definition}[BV Structure Theorem] A function $f \in L^1(U)$ has bounded variation in $U$ if
Then there exist a signed Radon measure $\mu$ on $U$  such that for all $\phi \in C_c^1\left(U ; \mathbb{R}^n\right)$, we have
$$
\int_U f \operatorname{div} \phi d x=-\int_U  \mu
$$
So we can  say that that the weak first partial derivatives of a BV function are Randon measures and we can therefore denote
$$\mu =D f.$$
\end{definition}

\begin{remark}
When not considered signed measures like in Evans book this notation is equivalent to:
$$|\mu|=\|Df\|\qquad \mu = [D f]:=\|D f\|\llcorner\sigma$$
\end{remark}

In the literature of Bounded deformation function we actually start with the structure theorem 




\begin{definition}[Bounded Deformation Function]
A function $f \in L^1(U)$ has bounded variation in $U$ if $$Ef_{ij}:=\frac{1}{2}(Df_{ij}+Df_{ji})\in M_1(U), \qquad \forall i,j $$
We write
$B D(U)$
to denote the space of functions of bounded deformation in $U$. 
Additionally, this is a Banach space for the norm
$$||f||_{BV(U)} = ||f||_{L^1(U)}+\sum_{i,j}|Ef_{ij}|(U)$$
\end{definition}
\begin{definition}[Equivalent Definition]
For every $\xi=\left(\xi^1, \ldots, \xi^n\right) \in \mathbf{R}^n$ let $D_{\xi}$ be the distributional derivative in the direction $\xi$ defined by $D_{\xi} v=\sum_j \xi^j D_j v$. For every function $f\in L^1(U)$ let us define the function $f^{\xi}: U \rightarrow \mathbf{R}$ by $f^{\xi}(x)=(f(x), \xi)$. Then $f \in B D(U)$ if 
$$
D_{\xi} f^{\xi}=(E f \xi, \xi)
$$
 is a bounded Radon measure on $U$ for every $\xi$ of the form $\xi_i+\xi_j, i, j=1, \ldots, n$.
 
Conversely, if $f \in B D(\Omega)$, then $D_{\xi} u^{\xi}$ is a bounded Radon measure on $U$ for every $\xi \in \mathbf{R}^n$.
\end{definition}


\section{Approximation and Compactness}

In both cases we can prove that the variation measure is lower semicontinuity and consequently we can have a local approximation with smooth functions.


\subsection{Bounded Variation}

\begin{theorem}[Lower semicontinuity]

Suppose $f_k \in B V(U)(k=1, \ldots)$ and
$$
f_k \rightarrow f \text { in } L_{\mathrm{loc}}^1(U)
$$
Then
$$
\|D f\|(U) \leq \liminf _{k \rightarrow \infty}\left\|D f_k\right\|(U)
$$
\end{theorem}
\begin{theorem}[Local approximation by smooth functions]

Assume $f\in BV(U)$. Then there exist functions $\left\{f_k\right\}_{k=1}^{\infty} \subset B V(U) \cap C^{\infty}(U)$ such that
\begin{itemize}
    \item[(i)] $f_k \rightarrow f$ in $L^1(U)$ and
    \item[(ii)] $\left\|D f_k\right\|(U) \rightarrow\|D f\|(U)$ as $k \rightarrow \infty$.
    \item [(iii)](theorem 5.4 I think) $\left\|D f_k\right\|  \rightharpoonup \|D f\|$ as $k \rightarrow \infty$.
\end{itemize}
\end{theorem}
\begin{theorem}[Compactness for BV functions]
 Let $U \subset \mathbb{R}^n$ be open and bounded, with Lipschitz boundary $\partial U$. Assume $\left\{f_k\right\}_{k=1}^{\infty}$ is a sequence in $B V(U)$ satisfying
$$
\sup _k\left\|f_k\right\|_{B V(U)}<\infty .
$$
Then there exists a subsequence $\left\{f_{k_j}\right\}_{j=1}^{\infty}$ and a function $f \in B V(U)$ such that
$$
f_{k_j} \rightarrow f \quad \text { in } L^1(U)
$$
as $j \rightarrow \infty$.
\end{theorem}
\begin{com}
the sequence comes from regularization so we can use the same theorem we use from bounded deformation.
I think it should the in $L^{1star}$
\end{com}
\subsection{Bounded Deformation}

Besides the norm topology defined by $\|f\|_{BD(U)}$ there are other 2 important topology defined in \cite{RogerBook} Chapter II, section 3.
\begin{definition}[Weak topology]
The weak topology in $BD(U)$ is the topology defined by the family of norms and seminorms:
$$\|f\|_{L^1(U)},\ \  \left| \int_U  \phi E_{ij} f \right| , \phi \in C^\infty_0(U).$$

in this topology, $f_k$ converges to $f$ if
$$\begin{cases} f_k\to f in L^1(U),\\
E_{ij}f_k \to E_{ij}f \ weakly\ in \ M_1(U).\end{cases}
$$
\end{definition}
So for every bounded sequence in BD(U) there is a weakly convergent subsequence.

\begin{definition}[Intermediate topology]
The intermediate topology can be defined by the distance 
$$\|f-g\|_{L^1(U)}+\left||Ef|(U)-|Eg|(U)\right|,$$

so $f_k$ converges to $f$ in the intermediate topology if $f_k$ converges in the weakly topology and additionally
$$
|Ef_k|(U)\to |Ef|(U)
$$
\end{definition}

\begin{theorem}
 Let $f\in BD(U)$ $f_\epsilon= \eta_\epsilon * f$ (some details need to be consider here for $U\neq \R^n$, but let it keep it simple) then when $\epsilon\to 0$ we have that
$$\left\{\begin{array}{l}
f_\epsilon \rightarrow f \text { in } L^1(U) \\
f_\epsilon(x)\to f^*(x) \text{ for all Lebesgue points},\\

E_{i j}f_\epsilon\to E_{i j}f \text { weakly in } M_1(U)
\end{array}\right.
$$
and
$$
\left\{\begin{array}{l}
\left|E_{i j}f_\epsilon\right|(V)\to\left|E_{i j}f\right|(V), \\
\text{for all } V\subset\bar{V}\subset U \text{such that } |E_{i j}f|(\partial V) = 0.
\end{array}\right.$$

Additionally we have that  if $f_k\in C^1$ converges to $f$ in $L^1$ and $$\int|Ef_k|dx < \infty$$
then $f_k$ converges weakly  to $f$ in $BD(U)$, and in particular $f\in BD(U)$
\end{theorem}
\begin{proof} check \cite{RogerBook}, page 147-148, and for the second part page 158/159\end{proof}

\begin{theorem}[More approximation of functions in $BD(U)$( Theorem 3.2,3.3 in \cite{RogerBook}]\ \ 
$C^\infty(\bar{U})$ is dense in the space $BD(U)$ endowed with the intermediate topology.

\end{theorem}

The following results are from the paper " Compactness and lower semicontinuity properties in $SBD(\Omega)$. (Read Section about the Jump set before)
\begin{theorem}
Let $\phi:[0,+\infty[\rightarrow[0,+\infty[$ be a non-decreasing function such that
$$
\lim _{t \rightarrow+\infty} \frac{\phi(t)}{t}=+\infty .
$$
Let $\left\{f_h\right\}$ be a sequence in $S B D(U)$ such that
$$\int_{U}\left|f_h\right| d x+\left|E^j f_h\right|(U)+\int_{U} \phi\left(\left|\mathcal{E} f_h\right|\right) d x+\mathcal{H}^{n-1}\left(J_{f_h}\right) \leq K$$
for some constant $K$ independent of $h$. Then there exist a subsequence, still denoted by $\left\{f_h\right\}$, and a function $f \in S B D(U)$ such that
\begin{align*}
f_h \rightarrow f & \text { strongly in } L_{\mathrm{loc}}^1\left(U ; \mathbf{R}^n\right) \\
\mathcal{E} f_h \rightharpoonup \mathcal{E} f & \text { weakly in } L^1\left(\Omega ; \mathbf{M}_{\mathrm{sym}}^{n \times n}\right) \\
E^j f_h \rightharpoonup E^j f & \text { weakly }{ }^* \text { in } \mathcal{M}_b\left(\Omega ; \mathbf{M}_{\mathrm{sym}}^{n \times n}\right), \\
\mathcal{H}^{n-1}\left(J_f\right) &\leq \liminf _{h \rightarrow \infty} \mathcal{H}^{n-1}\left(J_{f_h}\right)
\end{align*}
\end{theorem}


\begin{corollary}
Let $\left\{f_h\right\}$ be a sequence in $S B D(\Omega)$ which satisfies the bounds of the previous theorem and converges in $L_{\mathrm{loc}}^1\left(U ; \mathbf{R}^n\right)$ to a function $f \in S B D(U)$. Then

\begin{align*}
\int_{\Omega} f(x, \mathcal{E} u) d x & \leq \liminf _{h \rightarrow \infty} \int_{\Omega} f\left(x, \mathcal{E} u_h\right) d x \\
\int_{J_u} g\left(x,\left(u^{+}-u^{-}\right) \odot \nu_u\right) d \mathcal{H}^{n-1} \\
& \leq \liminf _{h \rightarrow \infty} \int_{J_{u_h}} g\left(x,\left(u_h^{+}-u_h^{-}\right) \odot \nu_{u_h}\right) d \mathcal{H}^{n-1}, \\
\mathcal{H}^{n-1}\left(J_u\right) & \leq \liminf _{h \rightarrow \infty} \mathcal{H}^{n-1}\left(J_{u_h}\right) .
\end{align*}
for 
\begin{enumerate}
    \item $f(x, \cdot)$ is convex and lower semicontinuous on $\mathbf{M}_{\mathrm{sym}}^{n \times n}$, 
    \item $f(\cdot, A)$ is measurable on $\Omega$,
    \item $g$ is lower semicontinuous on $\bar{\Omega} \times \mathbf{M}_{\mathrm{sym}}^{n \times n}$,
    \item $g(x, \cdot)$ is convex and positively 1-homogeneous on $\mathbf{M}_{\mathrm{sym}}^{n \times n}$.
\end{enumerate}
\end{corollary}

More compactness results in the GSBD section.
\section{Trace Theorems}

On this section we will always consider that $U$ is open and bounded, with a Lipschitz boundary $\partial U$. This means that $\partial U$  is locally a graph of a Lipschitz continuous function so there exists an outer unit normal vector $\nu$, $\mathcal{H}^{n-1}$ almost everywhere on $\partial U$

\subsection{Bounded Variation}
\begin{theorem}[Traces of BV functions]
 Assume $U$ is open and bounded, with $\partial U$ Lipschitz continuous. There exists a bounded linear mapping
$$
T: B V(U) \rightarrow L^1\left(\partial U ; \mathcal{H}^{n-1}\right)
$$
such that
$$
\int_U f \operatorname{div} \phi d x=-\int_U \phi \cdot d[D f]+\int_{\partial U}(\phi \cdot \nu) T f d \mathcal{H}^{n-1}
$$
for all $f \in B V(U)$ and $\phi \in C^1\left(\mathbb{R}^n ; \mathbb{R}^n\right)$ and that
$$Tf = f\qquad \CH^{n-1} \text{a.e. in} \partial U$$
for all $f\in BV(U)\cap C(\bar{U})$
\end{theorem}

\begin{theorem}[Extension of BV functions]
 Assume $U \subset \mathbb{R}^n$ is open and bounded, with $\partial U$ Lipschitz continuous. Let $f^1 \in B V(U)$, $f^2 \in B V\left(\mathbb{R}^n-\bar{U}\right)$
Define
$$
\bar{f}(x):= \begin{cases}f^1(x) & x \in U \\ f^2(x) & x \in \mathbb{R}^n-\bar{U}\end{cases}
$$
Then
$\bar{f} \in B V\left(\mathbb{R}^n\right)$
and
$$
\|D \bar{f}\|\left(\mathbb{R}^n\right)=\left\|D f^1\right\|(U)+\left\|D f^2\right\|\left(\mathbb{R}^n-\bar{U}\right)+\int_{\partial U}\left|T f^1-T f^2\right| d \mathcal{H}{ }^{n-1} .
$$
In particular $$\tilde{f}:= \begin{cases}f & \text { on } U \\ 0 & \text { on } \mathbb{R}^n-U\end{cases}$$
belongs to $BV(\R^n)$
\end{theorem}
\begin{com}
Exactly same result in BD, Should join them in the future. Actually in the future it should be just focus on BD since $BV \subset BD$ with some remarks on the things that do not work for BD
\end{com}

\begin{theorem}[Local Properties of traces]
Assume $U$ is open, bounded, with $\partial U$ Lipschitz continuous. Suppose also $f \in B V(U)$. Then for $\mathcal{H}^{n-1}-$ a.e. $x \in \partial U$,
$$
\lim _{r \rightarrow 0} \dashint_{B(x, r) \cap U}|f-T f(x)| d y=0,
$$
and so
$$
T f(x)=\lim _{r \rightarrow 0} \dashint_{B(x, r) \cap U} f d y= \tilde{f}^*(x).
$$
\end{theorem}
\begin{question}
Not sure there are similar results on BD
\end{question}
\subsection{Bounded Deformation}
On \cite{RogerPaper} they actually consider $C^1$ boundaries however all the results will certainly hold for Lipschitz boundary $\CH^{n-1} a.e$, so we will formulate them more similar to how Evans does. 
Additionally, the main difference is that in this case we can't retreive  the classic Green's formula since we don't know that $Df$ exists, but we can still get a generalized Green's Formula depending on the energy that is enough to prove  uniqueness and  linearity of the operator.
\begin{question}
Does  generalized Green's formula implies the classic one? I'm sure it must have.(maybe add the direct relation)
\end{question}

\begin{theorem}[Traces of BD functions]
 Assume $U$ is open and bounded, with $\partial U$ Lipschitz continuous. There exists a bounded linear mapping
$$
T: B V(U) \rightarrow L^1\left(\partial U ; \mathcal{H}^{n-1}\right)
$$
such that
$$
\int_U \left(f_j \phi_{,i}+f_i\phi_{,j}\right) d x=-2\int_U \phi \cdot E_{ij}f +\int_{\partial U}\phi(T f_i\nu_j+Tf_j \nu_i) d \mathcal{H}^{n-1}
$$
for all $i,j$, $f \in B D(U)$ and $\phi \in C^1\left(\mathbb{R}^n ; \mathbb{R}\right)$ and that
$$Tf = f\qquad \CH^{n-1} \text{a.e. in } \partial U$$
for all $f\in BV(U)\cap C(\bar{U})$.
\end{theorem}
We can also rewrite the green's formula in a way more similar to the Bounded variation. Given a smooth symmetric matrix-valued function $\varphi=\left\{\phi_{i j}\right\}$ with compact support in $\mathbf{R}^n$, we have that
$$\int_{U}(f, \operatorname{div} \phi) d x=-\sum_{i, j=1}^n \int_{U} \phi_{ij} E_{ij}f+\int_{\partial U}\left(Tf \odot \nu_{U}, \phi\right) d \CH^{n-1},$$ where $\nu_{U}$ is the outer unit normal to $\partial U$ and $a \odot b=\frac{1}{2}(a \otimes b+b \otimes a)$.
\begin{remark}
The trace operator is also continuous for the intermediate topology  but not for the weak topology.Additionally, for any $f$ in $B D(U)$, there exists a sequence $f_k \in \mathscr{C}(U)^n \cap L^1(U)$, $E_{ij}f_k\in L^1(U)$ such that
$$
T\left(f_k\right)=T(f), \text { for all } k \text {, }
$$
$$f_k \rightarrow f,\ in \ the \ intermediate\ topology$$
\end{remark}

\begin{theorem}[Extension of BD functions]
 Assume $U \subset \mathbb{R}^n$ is open and bounded, with $\partial U$ Lipschitz continuous. Let $f^1 \in B D(U)$, $f^2 \in B D\left(\mathbb{R}^n-\bar{U}\right)$
Define
$$
\bar{f}(x):= \begin{cases}f^1(x) & x \in U \\ f^2(x) & x \in \mathbb{R}^n-\bar{U}\end{cases}
$$
Then
$\bar{f} \in B D\left(\mathbb{R}^n\right)$
and
$$
\|E_{ij} \bar{f}\|\left(\mathbb{R}^n\right)=\left\|E_{ij} (f^1)\right\|(U)+\left\|E_{ij}( f^2)\right\|\left(\mathbb{R}^n-\bar{U}\right)+\int_{\partial U}\left|(T (f^1_i-f^2_i)\nu_j+T(f^1_j-f^2_j) \nu_i)\right| d \mathcal{H}{ }^{n-1} .
$$
In particular $$\tilde{f}:= \begin{cases}f & \text { on } U \\ 0 & \text { on } \mathbb{R}^n-U\end{cases}$$
belongs to $BD(\R^n)$
\end{theorem}

\begin{theorem}
Let $f$ in $BD(f)$ and let $\CM$ be a $C^1$ hypersurface, then bothd one-sided limits exist for $\CH^{n-1}$-almost every $x\in \CM$ and satisfy
$$E uf\left\llcorner M=\left(f^{+}-f^{-}\right) \odot v_M \mathscr{H}^{n-1}\llcorner M\right.$$
\end{theorem}
\begin{question}
Not 100\% sure about this theorem. Need to reconfirm if necessary. page 152 \cite{RogerBook}.
\end{question}

\section{Sobolev embedding}
\subsection{Bounded Variation}

\begin{theorem}[Theorem 5.10 Inequalities for BV functions]\hspace{1cm}

\begin{enumerate}[label=(\roman*)]
\item There exists a constant $C_1$ such that 
$$
\|f\|_{L^{1{ }^*}\left(\mathbb{R}^n\right)} \leq C_1\|D f\|\left(\mathbb{R}^n\right)
$$
for all $f \in B V\left(\mathbb{R}^n\right)$, where
$$
1^*=\frac{n}{n-1} .
$$
\item There exists a constant $C_2$ such that
$$
\left\|f-(f)_{x, r}\right\|_{L^{1 *}(B(x, r))} \leq C_2\|D f\|\left(B^0(x, r)\right)
$$
for all balls $B(x, r) \subset \mathbb{R}^n$ and $f \in B V_{\text {loc }}\left(\mathbb{R}^n\right)$, where
$$
(f)_{x, r}:=f_{B(x, r)} f d y .
$$

\item For each $0<\alpha \leq 1$, there exists a constant $C_3(\alpha)$ such that
$$
\|f\|_{L^{1^*}(B(x, r))} \leq C_3(\alpha)\|D f\|\left(B^0(x, r)\right)
$$
for all $B(x, r) \subseteq \mathbb{R}^n$ and all $f \in B V_{\text {loc }}\left(\mathbb{R}^n\right)$ satisfying $$ \frac{\mathcal{L}^n(B(x, r) \cap\{f=0\})}{\mathcal{L}^n(B(x, r))} \geq \alpha \text {. if } f \neq 0 \text { if } f(t) $$
\end{enumerate}
\end{theorem}
\subsection{Bounded Deformation}
\begin{theorem}[Theorem 2.2 Sobolev embedding of BD functions (Proposition 1.2 in the paper) ]\hspace{1cm}

\begin{enumerate}[label=(\roman*)]
\item There exists a constant $C_1$ such that 
$$
\|f\|_{L^{1{ }^*}\left(U\right)} \leq C_1 \sum_ {i,j} \|E_{ij}f\|\left(\R^n\right)\leq C\|f\|_{BD(U)}
$$
for all $f \in B D(U)$, where
$$
1^*=\frac{n}{n-1} .
$$

\item For every $f \in B D(U)$, there exists $R(f)$ such that $ER=0$  and a constant $C_2$ such that
$$
\|f-R(f)\|_{BD(U)} \leq C_2 \sum_{i, j=1}^N\left\|e_{i j}(f)\right\|_{M_1(U)}
$$
\end{enumerate}
\end{theorem}

\begin{theorem}[THEOREM 2.4. (Compactness Theorem)]
The injection of $B D(\Omega)$ into $L^p(\Omega)^n$ is compact, for all $p$ such that $1 \leq p<n^*=n /(n-1)$.
\end{theorem}
A easy way to prove this result is the fact that  that BD(U) is a dual space. (Proposition 2.5, \cite{RogerPaper}

\begin{theorem}[Poincare's inequality  BD functions]
Let $f \in B D\left(\mathbf{R}^n\right), n \geqq 2$, and for any $x \in \mathbf{R}^n$ and $r>0$ define
$$
\beta_{r}(x)=-\frac{1}{2 \omega_n} \int_{\{|y-x| \geqq r\}} \frac{\Gamma(y-x)}{|y-x|^{n+2}} d E f(y) .
$$
Then, for suitable vectors $d_\epsilon(x) \in \mathbf{R}^n$,
$$
\int_{B_\epsilon(x)}\left|f(y)-d_\epsilon(x)-\beta_\epsilon(x)(y-x)\right| d y \leqq c_n Q|E u|\left(B_{r}(x)\right),
$$
where $c_n$ depends only on $n$.
\end{theorem}
\section{One Dimension Representation}
\subsection{Functions of one variable}

Is trivially to see that for  a function of one variable we have that $$BV(\R)=BD(\R)\qquad Df=Ef,$$

And additionally, for $f\in BD(a,b)$ we have that
$$Df(a,b)=Ef(a,b)=\operatorname{ess} V_a^b f:=\sup \left\{\sum_{j=1}^m\left|f\left(t_{j+1}\right)-f\left(t_j\right)\right|\right\}
$$
where the supremum taken over all finite partitions $\left\{a<t_1<\cdots<t_{m+1}<b\right\}$ such that each $t_i$ is a point of approximate continuity of $f$.

Equivalently we can also $$\operatorname{ess} V_a^b f = V_a^b f^*=\sup \left\{\sum_{j=1}^m\left|f^*\left(t_{j+1}\right)-f^*\left(t_j\right)\right|\right\}$$
where the supremum taken over all finite partitions $\left\{a<t_1<\cdots<t_{m+1}<b\right\}$ such that each $t_i$ is a point of Lebesgue continuity of $f$.

Additionally, we can extend this to a non negative Randon measure to any open subset $U_1$ of $\R$, $ess Vf=Vf*$.

To use this results we can define the the one dimensional function for
$$f_y^{\xi}(t)=f^{\xi}(y+t \xi)=(f(y+t \xi), \xi) \quad \forall t \in U_y^{\xi} .$$

for  each multi variable function $f\in L^1(U)$, and  every $\xi,y\in\R^n$, and the respective measure $V f_y^{*\xi}$


 
\subsection{Bounded Variation}
\begin{theorem} [THEOREM $5.22$ Essential variation on lines]. Assume that $f \in L_{\text {loc }}^1\left(\mathbb{R}^n\right)$. Then $f \in B V_{\mathrm{loc}}\left(\mathbb{R}^n\right)$ if and only if
$$
\int_K \operatorname{ess} V_a^b f_y^{e_i+ e_j} d y<\infty \quad(i,j=1, \ldots, n)
$$
for all $-\infty<a<b<\infty$ and all compact sets $K \subset \mathbb{R}^{n-1}$.
\end{theorem}
\begin{com}
I would assume that we can have something very similar for a set $U$. Also this is not exactly the same result since evans just consider functions whit co-domain $\R$. but this makes sense in my opinion.
\end{com}
\subsection{Bounded Deformation}


\begin{proposition}[Proposition 3.2 Ambrosio's]
 Let $f \in B D(U)$ and let $\xi \in \mathbf{R}^n$ with $\xi \neq 0$. Then the following two conditions hold for $\CH^{n-1}$-almost every y $\in U^{\xi}$ :\begin{enumerate}
\item ${f^*}_y^{\xi}$ is defined and coincides with $f_y^\xi \ \CL ^1$-almost everywhere in $U_y^{\xi}$;
\item $f_y^{\xi} \in B V\left(U_y^{\xi}\right)$ and the measures $|D f_y^{\xi}|$ and $V f^*_{y^{\xi}}$ coincide on $U_y^{\xi}$.
\end{enumerate}

Moreover,(review notation)
$$(Ef \xi, \xi)=\int_{U^{\xi}} \|D f_y^{\xi}\| d \CH^{n-1}(y), \quad|(E f \xi, \xi)|=\int_{U^{\xi}}\left|D f_y^{\xi}\right| d \CH^{n-1}(y)$$
as measures in $U$. Conversely, let $f \in L^1\left(\Omega ; \mathbf{R}^n\right)$ and let $\xi_1, \ldots, \xi_n$ be a basis of $\mathbf{R}^n$. Assume that for every $\xi$ of the form $\xi_i+\xi_j$,
$$
\begin{aligned}
&f_y^{\xi} \in B V\left(U_y^{\xi}\right) \text { for } \CH^{n-1} \text {-a.e. } y \in U^{\xi}, \\
&\int_{U^{\epsilon}}\left|D f_y^{\xi}\right|\left(U_y^{\xi}\right) d \CH^{n-1}(y)<+\infty .
\end{aligned}
$$
Then $f \in B D(U)$.
\end{proposition}


\section{Lebesgue discontinuity points ($S_f$) and Jump points ($J_f$)}
We already know that for $f\ \CL^n-$measurable we have that $$\CL^n(S_f)=\CL^n(J_f)=0$$. However when we assume more regularity in the function $f$ we get more information on the set.
\subsection{Bounded Variation Functions}

\begin{theorem}[Approximating by hypersurfaces] There exist countably many $C^1$-hypersurfaces $\left\{M_k\right\}_{k=1}^{\infty}$ such that
$$
\mathcal{H}^{n-1}\left(J-\bigcup_{k=1}^{\infty} M_k\right)=0.
$$
Additionally for each $M_k$ we have a unique $C^1$ outer vector $\nu_k$ and we can define $$\nu_f(x)=\nu_k (x),\quad x\in M_k,$$
so $\nu_f(x)$ is continuous for $\CH^{n-1} a.e.$ in $J$ 
\end{theorem}
\begin{theorem}[Approximate lim sup and lim inf]. We have
$$
-\infty<\lambda(x) \leq \mu(x) <+\infty
$$

for $\mathcal{H}^{n-1}$-a.e. $x \in \mathbb{R}^n$.
Which implies that the function $f$ has one sided Lebesgue limit $\mathcal{H}^{n-1}$-a.e. and $\CH^{n-1}(S_f-J_f)=0$ (do you agree?????????????????)
\end{theorem}
Now define  $F(x):=\frac{\lambda(x)+\mu(x)}{2},$ to be the average of the ap $\limsup f$ and ap $\liminf f$. 
\begin{theorem}[Fine properties of BV functions] Assume $f \in B V\left(\mathbb{R}^n\right)$.
\begin{enumerate}[label=(\roman*)]
    \item  Then for $\mathcal{H}^{n-1}$-a.e. $x \in \mathbb{R}^n-J$, we have 
$$
\lim _{r \rightarrow 0} \dashint_{B(x, r)}|f-F(x)|^{\frac{n}{n-1}} d y=0 .
$$
\item Furthermore, for $\mathcal{H}^{n-1}$-a.e. $x \in J$, there exists a unit vector $\nu_f=$ $\nu_f(x)$ such that
$$
\lim _{r \rightarrow 0} \dashint_{B(x, r) \cap H_{\nu_f}^{-}}|f-\mu(x)|^{\frac{n}{n-1}} d y=0
$$
and
$$
\lim _{r \rightarrow 0} \dashint_{B(x, r) \cap H_{\nu_f}^{+}}|f-\lambda(x)|^{\frac{n}{n-1}} d y=0 .
$$
\end{enumerate}
\end{theorem}
Thus we see that for $\mathcal{H}^{n-1}$-a.e. $x \in J, f$ has a "measure theoretic jump" across the hyperplane $H_{\nu(x)}$.
\begin{remark}
This theorem also conclude that for  $\mathcal{H}^{n-1}$-a.e. $x \in J$ there exists one side Lebesgue limits and $f^+=\mu(x),f^-=\lambda(x)$
\end{remark}

\begin{theorem}(BV and mollifiers)
\begin{enumerate}
\item If $f \in B V\left(\mathbb{R}^n\right)$, then
$$f^*(x):=\lim _{r \rightarrow 0}(f)_{x, r}=F(x)$$
exists for$\mathcal{H}^{n-1} \text {-a.e. } x \in \mathbb{R}^n$.

\item Furthermore, if $\eta_\epsilon$ is the standard mollifier and $f^\epsilon:=\eta_\epsilon * f$, then
$$
f^*(x)=\lim _{\epsilon \rightarrow 0} f^\epsilon(x)
$$
for $\mathcal{H}^{n-1}$-a.e. $x \in \mathbb{R}^n$.
\end{enumerate}
\end{theorem}

\begin{question}
What can we say about $S_f$ though? In FEDERER, H.: Geometric Measure Theory it looks like they prove that eve $S_f$ can be covered by a sequence of hypersuface. which makes will be immediate from the conclusion of Theorem 9.2 correct?
\end{question}
\begin{com}
from my understanding  in evans gariepy they prove that $J_u$ is $H^{n-1}$ rectifiable and that $J_u=S_u H^{n-1}$ a.e. so $S_u$ is also $H^{n-1}$ rectifiable. I have seen in other places that you can also prove directly that $S_u$ is $H^{n-1}$ rectifiable so also $J_u$.

Additionally this also implies that when we split the measure $Df$ we can use the $H^{n-1}$ part using $J_f$ of $S_f$
\end{com}


\subsection{Bounded Deformation Functions}
For bounded deformation not all this results can be proved. In fact we can prove that $S_f$ is $H^{n-1}$ rectifiable but is still an open problem ( at least when this paper was written) that $\CH^{n-1}(S_f-J_f)=0$.

So let consider the set $\Theta_f$ first defined in Korn Phd Thesis,
$$\Theta_f:=\left\{x \in U: \limsup _{r\rightarrow 0^{+}} \frac{|E f|\left(B_{r}(x)\right)}{r^{n-1}}>0\right\}$$
\begin{proposition}
For every $f \in B D(\Omega)$, the jump set $J_f$ is a countably $\left(\mathscr{H}^{n-1}, n-1\right)$-rectifiable Borel set. Moreover, $J_f\subset\Theta_f$ and $\CH^{n-1}\left(\Theta_u \backslash J_u\right)=0.$ 
\end{proposition}
\begin{proof}
They prove first that $\Theta_f$ is rectifiable and then that $\CH^{n-1}\left(\Theta_u \backslash J_u\right)=0.$ The proves use a lot of resultes form Federer which I can't really understand the notation. however the techniques look very similar to the ones with BV, so maybe in the future I could just prove them directly.

Additionally, After knowing that $J_f$ is rectifiable then the conclusion that both definition work come from the lemma 2.2 in \cite{RogerBook}
\end{proof}

\begin{question}
 does this work for our $J_f$ also? I need to re-look into the proof. I think this $\Theta$ set will be deeply related to the measure theoretic boundary...
\end{question}

Even that $\CH^{n-1}(S_f-J_f)=0$ is not known it was proven that $\CH^{n-1+\epsilon}(S_f-J_f)=0$ by Kohn and 
\begin{theorem}
 If $u, v \in B D(\Omega)$, then $|E v|\left(S_u \backslash J_u\right)=0$. In other words, $|E v|-a l-$ most every point $x \in \Omega$ is either a Lebesgue point or a jump point of $u$.
\end{theorem}
\begin{com}
Look into remark 6.3 in more details.
\end{com}
From "Which special functions of bounded deformation
have bounded variation?"
\begin{theorem}
Theorem 4.1. If $u \in S B D^p(\Omega)$ for some $p>1$, with $\Omega \subset \mathbb{R}^n$ open, then $\mathcal{H}^{n-1}\left(S_u \backslash J_u\right)=0$
\end{theorem}

\section{Lebesgue decomposition of Df and Ef- Structure Theorem}

\subsection{Bounded Variation}
\begin{definition} Let $f \in B V(\Omega)$. Then we can decompose the measure $Df$ into
$$D f=D _{ac}f+D_jf +D_cf$$
such that:
\begin{enumerate}
\item The approximate differential of $f$ is the density $\nabla f$ of $E_{ac} f$ with respect to $\CL^n$, i.e., $$D_{ac} f=\nabla f \mathscr{L}^n.$$
\item The jump part of $D f$ is the restriction $D_j f$ of $D_s f$ to the jump set $D_f$, i.e., $$D_f u:=D_s f\left\llcorner J_f\right.$$
\item The Cantor part of $D f$ is the restriction $D_c f$ of $D_s f$ to the complement of $J_f$, i.e., $$D_c u:=D_s f\left\llcorner\left(U \backslash J_f\right)\right.$$
\end{enumerate}
\end{definition}
Since the set $J_f$ if countably $\CH^{n-1}-$rectifiable we have that 
$$D f_j=\left(u^{+}-u^{-}\right) \otimes \nu_f \mathscr{H}^{n-1}\llcorner J_f.$$

Additionally can also be proven that this is the only $n-1$ dimensional part of $Df$ since the cantor part $D_c$ vanishes for all Borel sets $B$ with $\mathscr{H}^{n-1}(B)<+\infty$.(Proven for BD in Ambrosio's Paper)

Since $S_f\backslash J_f$ is $\CH^{n-1}$ negligible we can also get $D_j$ by restrict to the bigger set $S_f$.


We say that $f$ is a special function of bounded variation in $U$ if and $D _cf=0$ in $U$, and  we write the set of this functions as $S B V(U)$. 
\subsection{Bounded Deformation}

\begin{definition} Let $f \in B D(U)$. Then we can decompose the measure $Ef$ into
$$E f=E _{ac}f+E_jf +E_cf$$
such that:
\begin{enumerate}
\item The approximate symmetric differential of $f$ is the density $\CE f$ of $E_{ac} f$ with respect to $\CL^n$, i.e., $$E_{ac} f=\CE f \mathscr{L}^n$$.
\item The jump part of $E f$ is the restriction $E_j f$ of $E_s f$ to the jump set $J_f$, i.e., $$E_f u:=E_s f\left\llcorner J_f\right.$$.
\item The Cantor part of $E f$ is the restriction $E_c f$ of $E_s f$ to the complement of $J_f$, i.e., $$E_c u:=E_s f\left\llcorner\left(U \backslash J_f\right)\right.$$
\end{enumerate}
\end{definition}

Since the set $J_f$ if countably $\CH^{n-1}-$rectifiable we have that 
$$E_j f=\left(u^{+}-u^{-}\right) \odot v_f \mathscr{H}^{n-1}\left\llcorner J_f\right.$$

Additionally can also be proven that this is the only $n-1$ dimensional part of $Ef$ since the cantor part $E_c$ vanishes for all Borel sets $B$ with $\mathscr{H}^{n-1}(B)<+\infty$.(Proven for BD in Ambrosio's Paper)

Since $\Theta_f\backslash J_f$ is $\CH^{n-1}$ negligible we can also get $E_j$ by restrict to the bigger set $\Theta_f$.

For the main structure theorem consider the set:

$$
J_f^{\xi}:=\left\{x \in J_f:\left(u^{+}(x)-u^{-}(x), \xi\right) \neq 0\right\} .
$$
that satisfies $\quad \mathscr{H}^{n-1}\left(J_u \backslash J_u^{\xi}\right)=0$ for $\mathscr{H}^{n-1}$-a.e. $\xi \in \mathbf{S}^{n-1}$.

\begin{theorem}
Structure Theorem 4.5. Let $f \in B D(U)$ and let $\xi \in \mathbf{R}^n$ with $\xi \neq 0$. Then
\begin{enumerate}
\item $\left( \CE f \xi, \xi\right)=\int_{U^{\xi}} \nabla f_y^{\xi} d \mathscr{H}^{n-1}(y),\left|\left(\CE d \xi, \xi\right)\right|=\int_{U^{\xi}}\left|\nabla f_y^{\xi}\right| d \mathscr{H}^{n-1}(y)$.

\item For $\mathscr{H}^{n-1}$-almost every $y \in U^{\xi}$, the functions $f_y^\zeta$ and $f_y^{*\xi}$ belong to $B V\left(U_y^{\xi}\right)$ and coincide $\mathscr{L}^1$-almost everywhere on $U_y^{\xi}$, the measures $\left|D f_y^{\xi}\right|$ and $V f_y^{*\xi}$ coincide on $U_y^{\xi}$, and $$(\CE f(y+t \xi) \xi, \xi)=\nabla f_y^{\xi}(t)=\left(f_y^{*\xi}\right)^{\prime}(t)$$ for $\mathscr{L}^1$-almost every $t \in U_y^{\xi}$.
\item $\left(E_j f \xi, \xi\right)=\int_{U^{\xi}} D_j f_y^{\xi} d \mathscr{H}^{n-1}(y), \quad\left|\left(E_j f \xi, \xi\right)\right|=\int_{U^{\xi}}\left|D_j f_y^{\xi}\right| d \mathscr{H} \mathscr{H}^{n-1}(y)$.
\item $\left(J_f^{\xi}\right)_y^{\xi}=J_{f_j^{\xi}}$ for $\mathscr{H}^{n-1}$-almost every $y \in U^{\xi}$
$$
\begin{aligned}
&\left(f^{+}(y+t \xi), \xi\right)=\left(f_y^{\xi}\right)^{+}(t)=\lim _{s \rightarrow t^{+}} f_y^{*\xi}(s), \\
&\left(f^{-}(y+t \xi), \xi\right)=\left(f_y^{\xi}\right)^{-}(t)=\lim _{s \rightarrow t^{-}} f_y^{*\xi}(s),
\end{aligned}
$$
and for every $t \in\left(J_f^{\xi}\right)_y^{\xi}$ where the normals to $J_f$ and $J_{f_y^{\xi}}$ are oriented so that $\left(v_f, \xi\right) \geq 0$ and $v_{f_j^{\xi}}=1$.
\item $\left(E_c f \xi, \xi\right)=\int_{U^{\xi}} D_fc u_y^{\xi} d \mathscr{H}^{n-1}(y),\left|\left(E_c f \xi, \xi\right)\right|=\int_{U^{\xi}}\left|D_c f_y^{\xi}\right| d \mathscr{H}^{n-1}(y)$.
\end{enumerate}
\end{theorem}


We say that $f$ is a special function of bounded Deformation in $U$ if and $E _cf=0$ in $U$, and  we write the set of this functions as $S B D(U)$. 

We can again find out if if a function is $SBD$ by looking at the functions $f_y^\xi$,i.e
\begin{proposition}
 Let $f \in B D(U)$ and let $\xi_1, \ldots, \xi_n$ be a basis of $\mathbf{R}^n$. Then the following three conditions are equivalent:
 \begin{enumerate}
 \item $f \in \operatorname{SBD}(U)$.
\item  For every $\xi=\xi_i+\xi_j$ with $1 \leqq i, j \leqq n$, we have $f_y^{\xi} \in S B V\left(U_y^{\xi}\right)$ for $\mathscr{H}^{n-1}$ almost every $y \in U \xi$.
\item The measure $\left|E_s f\right|$ is concentrated on a Borel set $B \subset U$ which is $\sigma$-finite with respect to $\mathscr{H}^{n-1}$.
\end{enumerate}
\end{proposition}

Lastly we can get a better representation of the jump set:
\begin{theorem}
Let $f \in B D(\Omega)$ and define the non-negative Borel measure $\lambda_f$ on $U$ as
$$
\lambda_f(B):=\kappa_n \int_{\mathbf{S}^{n-1}} \lambda_f^{\xi}(B) d \mathscr{H}^{n-1}(\xi) \quad \forall B \in \mathscr{B}(U),
$$
where $\kappa_n:=\left(2 \omega_{n-1}\right)^{-1}$ and for every $\xi \in \mathbf{S}^{n-1}$ and
$$
\lambda_f^{\xi}(B):=\int_{U^{\xi}} \mathscr{H}^0\left(J_{f_y^{\xi}} \cap B_y^{\xi}\right) d \mathscr{H}^{n-1}(y) \quad \forall B \in \mathscr{B}(U)
$$
then
$$
\lambda_f^\xi(B)=\int_{J_f^{\xi} \cap B}\left|\left(\mu_f, \xi\right)\right| d \mathscr{H}^{n-1} \quad \forall B \in \mathscr{B}(U),
$$
where $\mu_f$ is the approximate unit normal to $J_f$. Moreover $$\lambda_f=\mathscr{H}^{n-1}\left\llcorner J_f\right.$$.
\end{theorem}
\begin{theorem}
We have $B V\left(\Omega ; \mathbf{R}^n\right) \cap S B D(\Omega)=S B V\left(\Omega ; \mathbf{R}^n\right)$. Moreover the inclusions $S B V\left(\Omega ; \mathbf{R}^n\right) \subseteq S B D(\Omega) \subseteq B D(\Omega)$ are strict.
\end{theorem}

\section{Approximate Differentiability/ Lp differentiable what should I focus on?}


\begin{definition}
DEFINITION 6.1. Let $f: \mathbb{R}^n \rightarrow \mathbb{R}^m$. We say $f$ is approximately differentiable at $x \in \mathbb{R}^n$ if there exists a linear mapping
$$
L: \mathbb{R}^n \rightarrow \mathbb{R}^m
$$
such that
$$
\operatorname{aplim}_{y \rightarrow x} \frac{|f(y)-f(x)-L(y-x)|}{|y-x|}=0
$$
\end{definition}
Additionally, for $g\in L^1$ then every Lebesgue point is actually a point of aproximatly continuous, so

$$\lim _{r \rightarrow 0^{+}} \frac{1}{r^n} \int_{B_(x,r)} \frac{|f(y)-f(x)-(L f(x), y-x)|}{|y-x|}dy=0$$ 
implies the approximate differentiable at $x$. ( the oposite implication does not hold, unless $f$ is locally bounded)
If $L$ exists and is uniqye we write $$L= ap\ Df$$

Additionally we recall that functions wich are approximately differentiable $\CL^n$ a.e. can be approximated by lipschitz functions (and $C^1$)
\subsection{Bounded Variation}
\begin{theorem}
THEOREM 6.1 (Differentiability for BV functions). Assume that $f \in B V_{\mathrm{loc}}\left(\mathbb{R}^n\right)$. Then for $\mathcal{L}^n$-a.e. $x \in \mathbb{R}^n$,
$$
\left(f_{B(x, r)}|f(y)-f(x)-\nabla f(x) \cdot(y-x)|^{1^*} d y\right)^{\frac{1}{1^*}}=o(r)
$$
-as $r \rightarrow 0$.
\end{theorem}
\begin{theorem}
Assume $f\in BV(U)$. Then f is approximately differentiable $\CL^n$ a.e. Additionally we have that
$$ap\ Df = Df\qquad \CL^n\text{ a.e.}$$
\end{theorem}
\subsection{Bounded Deformation}
\begin{theorem}
Theorem 4.3. Let $f \in B D(U)$. For $\mathscr{L}^n$-almost every $x \in U$,
$$
\lim _{r \rightarrow 0^{+}} \frac{1}{\varrho^n} \int_{B(x,r)} \frac{|(f(y)-f(x)-\CE f(x)(y-x), y-x)|}{|y-x|^2} d y=0
$$
where $\CE f$ is the density of E to respect to the Lebesgue measure.
\end{theorem}

\begin{theorem}
Theorem 7.4. Let $f \in B D\left(\mathbf{R}^n\right)$. Then for $\mathscr{L}^n$-almost every $x \in \mathbf{R}^n$ there exists an $n \times n$ matrix ap $Df$ such that
$$
\lim _{r \rightarrow 0^{+}} \frac{1}{r^n} \int_{B(x,r)} \frac{|f(y)-f(x)-ap\ Df (x)(y-x)|}{r} d y=0 .
$$
In particular, $f$ is approximately differentiable $\mathscr{L}^n$-almost everywhere in $\mathbf{R}^n$. Moreover the function ap $Df$ satisfies the weak $L^1$ estimate
$$
\mathscr{L}^n\left(\left\{x \in \mathbf{R}^n:|ap\  Df|>t\right\}\right) \leq \frac{c(n)}{t}|E u|\left(\mathbf{R}^n\right) \quad \forall t>0,
$$
where $c(n)$ is a constant which depends only on $n$.
\end{theorem}


\section{Regularity theorems}
I didn't see this result directly on BV functions but it kinda sounds important
\begin{theorem}[Regularity]
\begin{enumerate}[label=(\roman*)]
\item If $f \in \mathscr{D}^{\prime}(U)$ and if each $e_{i j}(f) \in M_1(U)$, then $f$ is a function in $=L^{n /(n-1)}(U)$
\begin{question}
I actually do not understand this prove, is it important?
\end{question}
\item If each $e_{ij}(f_k)$ converges in $\mathscr{D}^{\prime}(U)$ there there exists $f\in\mathscr{D}^{\prime}(U)$ such that $e_{ij}=e_{ij}(f)$,  $i,j=1,\ldots,n$
\end{enumerate}
\end{theorem}
\section{Generalized Bounded Deformation}

It looks like considering just BD functions creates difficulties on obtain a priori bounds of the minimizing sequence.
To solve that problem in the BV settings the following functions were introduced:
\begin{definition}
Let $f: U \to \R^n$ be  a $\CL^n$ measurable function. We say that $f\in GSBV$ if for all $\phi(f)\in C^1(\R^n:\R^n)$ such that $\nabla\phi$ as compact support, then $\phi(f)\in BV_{loc}(U:\R^n)$
\end{definition}

However the same strategy goes not work for BD functions because in general for $f\in BD$ we do not have that $\phi(f)\ in BD$.

\begin{definition}
 The space $G B D(U)$ of generalised functions of bounded deformation is the space of all $\mathcal{L}^n$-measurable functions $f: U \rightarrow \mathbb{R}^n$ with the following property: there exists $\lambda \in \mathcal{M}_b^{+}(\Omega)$ such that the following equivalent conditions hold for every $\xi \in \mathbb{S}^{n-1}:$
\begin{enumerate}
     \item for every $\tau \in \mathcal{T}$ the partial derivative $D_{\xi}(\tau(u \cdot \xi))$ belongs to $\mathcal{M}_b(\Omega)$ and its total variation satisfies
$$
\left|D_{\xi}(\tau(u \cdot \xi))\right|(B) \leq \lambda(B)
$$
for every Borel set $B \subset \Omega$;
\item for $\mathcal{H}^{n-1}$-a.e. $y \in \Pi \Pi^{\xi}$ the function $\hat{u}_y^{\xi}:=u_y^{\xi} \cdot \xi$ belongs to $B V_{l o c}\left(\Omega_y^{\xi}\right)$ and
$$
\int_{\Pi \xi}\left(\left|D \hat{u}_y^{\xi}\right|\left(B_y^{\xi} \backslash J_{\hat{u}_y^{\xi}}^1\right)+\mathcal{H}^0\left(B_y^{\xi} \cap J_{\hat{u}_y^{\xi}}^1\right)\right) d \mathcal{H}^{n-1}(y) \leq \lambda(B)
$$
for every Borel set $B \subset \Omega$.
\end{enumerate}
This second property is equivalent to \\


$\left(ii^{\prime}\right)$ for $\mathcal{H}^{n-1}$-a.e. $y \in \Pi^{\xi}$ the function $\hat{u}_y^{\xi}:=u_y^{\xi} \cdot \xi$ belongs to $G B V\left(\Omega_y^{\xi}\right)$ and $$\int_{\Pi \xi}\left(\left|D\left(\sigma_a\left(\hat{u}_y^{\xi}\right)\right)\right|\left(B_y^{\xi} \backslash J_{\sigma_a\left(\hat{u}_y^{\xi}\right)}^1\right)+\mathcal{H}^0\left(B_y^{\xi} \cap J_{\sigma_a\left(\hat{u}_y^{\xi}\right)}^1\right)\right) d \mathcal{H}^{n-1}(y) \leq \lambda(B)$$ for every Borel set $B \subset \Omega$ and for every $a>0$,
where $\sigma_a$ be the truncation function defined by $\sigma_a(t)=-a$ for $t \leq-a, \sigma_a(t)=t$ for $-a \leq t \leq a$, and $\sigma_a(t)=a$ for $t \geq a$. 
\end{definition}


\begin{definition}
 The space $G S B D(\Omega)$ of generalised special functions of bounded deformation is the set of all functions $u \in G B D(\Omega)$ such that for every $\xi \in \mathbb{S}^{n-1}$ and for $\mathcal{H}^{n-1}$-a.e. $y \in \Pi^{\xi}$ the function $\hat{u}_y^{\xi}:=u_y^{\xi} \cdot \xi$ belongs to $S B V_{l o c}\left(\Omega_y^{\xi}\right)$ 
\end{definition}

Another important measure related to $f$ is $$\hat{\mu}^{\xi}(B):=\int_{\Pi \xi} \hat{\mu}_y^{\xi}\left(B_y^{\xi}\right) d \mathcal{H}^{n-1}(y)$$
with
$$\hat{\mu}_y^{\xi}(B):=\left|D \hat{u}_y^{\xi}\right|\left(B \backslash J_{\hat{u}_y^{\xi}}^1\right)+\mathcal{H}^0\left(B \cap J_{\hat{u}_y^{\xi}}^1\right)$$
and
$$\hat{\mu}_u(B):=\sup _k \sup \sum_{i=1}^k \hat{\mu}_u^{\xi_i}\left(B_i\right)$$
Some important properties:
\begin{enumerate}
    \item $BD\subset GBD$, $SBD\subset GSBD$, with $\lambda(B)= |Ef|(B)$
    \item $\hat{\mu}^{\xi}(B)<\lambda<(B)$ and $\hat{\mu}_{f}(B)<\lambda<(B)$
    \item $\hat{\mu}_u^{\xi}(U) \leq \liminf _{k \rightarrow \infty} \hat{\mu}_{u_k}^{\xi}(U)$
    \item some simillar Trace Theorem
    \item One side limits for any $C^1$ hyperplane
    \item $\Omega_f$ is also countably rectifiable and $\CH^{n-1}(\Omega_f-S_f)=0$
    \item $\left(J_u^{\xi}\right)_y^{\xi}=J_{\hat{u}_y^{\xi}}$
    \item it has an appproximate symmetric gradient $L^n$ almost everywhere
    \item Some Comcpactness results I do not completely understand:  Similar result  for sbd functions. in "Compactness and lower semicontinuity properties
in SBD()"
    Theorem 11.3. Let $u_k$ be a sequence in $G S B D(\Omega)$. Suppose that there exist a constant $M \in \mathbb{R}^{+}$and two increasing continuous functions $\psi_0: \mathbb{R}^{+} \rightarrow \mathbb{R}^{+}$and $\psi_1: \mathbb{R}^{+} \rightarrow \mathbb{R}^{+}$, with
$$
\lim _{s \rightarrow+\infty} \psi_0(s)=+\infty \quad \text { and } \quad \lim _{s \rightarrow+\infty} \frac{\psi_1(s)}{s}=+\infty,
$$
such that
$$
\int_{\Omega} \psi_0\left(\left|u_k\right|\right) d x+\int_{\Omega} \psi_1\left(\left|\mathcal{E} u_k\right|\right) d x+\mathcal{H}^{n-1}\left(J_{u_k}\right) \leq M
$$
for every $k$. Then there exist a subsequence, still denoted by $u_k$, and a function $u \in$ $\operatorname{GSBD}(\Omega)$, such that
$$
\begin{gathered}
u_k \rightarrow u \quad \text { pointwise } \mathcal{L}^n \text {-a.e. on } \Omega, \\
\mathcal{E} u_k \rightarrow \mathcal{E} u \quad \text { weakly in } L^1\left(\Omega ; \mathbb{M}_{\text {sym }}^{n \times n}\right), \\
\mathcal{H}^{n-1}\left(J_u\right) \leq \liminf _{k \rightarrow \infty} \mathcal{H}^{n-1}\left(J_{u_k}\right) .
\end{gathered}
$$
If, in addition, (11.3) holds, then $u_k \in L^1\left(\Omega ; \mathbb{R}^n\right)$ for every $k, u \in L^1\left(\Omega ; \mathbb{R}^n\right)$, and the subsequence converges strongly in $L^1\left(\Omega ; \mathbb{R}^n\right)$.
\end{enumerate}

One more important compactness result shows up in "Compactness and lower semicontinuity in GSBD"

\begin{theorem}
Let \( \phi: \mathbb{R}^{+} \rightarrow \mathbb{R}^{+} \)be a non-decreasing function with
\[
\lim _{t \rightarrow+\infty} \frac{\phi(t)}{t}=+\infty,
\]
and let \( \left(u_h\right)_h \) be a sequence in \( G S B D(\Omega) \) such that
\[
\int_{\Omega} \phi\left(\left|e\left(u_h\right)\right|\right) \mathrm{d} x+\mathcal{H}^{n-1}\left(J_{u_h}\right)<M
\]
for some constant \( M \) independent of \( h \). Then there exists a subsequence, still denoted by \( \left(u_h\right)_h \), such that
\[
A:=\left\{x \in \Omega:\left|u_h(x)\right| \rightarrow+\infty\right\}
\]
has finite perimeter, and \( u \in G S B D(\Omega) \) with \( u=0 \) on A for which
\[
\begin{aligned}
&u_h \rightarrow u \quad \mathcal{L}^n \text {-a.e. in } \Omega \backslash A, \\
&e\left(u_h\right) \rightarrow e(u) \quad \text { in } L^1\left(\Omega \backslash A ; \mathbb{M}_{\text {sym }}^{n \times n}\right), \\
&\mathcal{H}^{n-1}\left(J_u \cup \partial^* A\right) \leq \liminf _{h \rightarrow \infty} \mathcal{H}^{n-1}\left(J_{u_h}\right) .
\end{aligned}
\]
\end{theorem}

We have again a Korn-Type inequality:
\begin{proposition}
 Let $0<\theta^{\prime \prime}<\theta^{\prime}<1, r>0$. Let $Q=(-r, r)^n, Q^{\prime}=\left(-\theta^{\prime} r, \theta^{\prime} r\right)^n, p \in[1, \infty), u \in$ $G S D^p(Q)$
1. There exists a set $\omega \subset Q^{\prime}$ and an affine function $a: \mathbb{R}^n \rightarrow \mathbb{R}^n$ with $e(a)=0$ such that
$$
|\omega| \leq c_* r \mathcal{H}^{n-1}\left(J_u\right)
$$
and
$$
\int_{Q^{\prime} \backslash \omega}|u-a|^{n p /(n-1)} \leq c_* r^{n(p-1) /(n-1)}\left(\int_Q|e(u)|^p d x\right)^{n /(n-1)}
$$
2. If additionally $p>1$ then there is $\bar{p}>0$ (depending on $p$ and $n$ ) such that, for a given mollifier $\rho_r \in C_c^{\infty}\left(B_{\left(\theta^{\prime}-\theta^{\prime \prime}\right) r}\right), \rho_r(x)=r^{-n} \rho_1(x / r)$, the function $v=u \chi_{Q^{\prime} \backslash \omega}+a \chi_\omega$ obeys
$$
\int_{Q^{\prime \prime}}\left|e\left(v * \rho_r\right)-e(u) * \rho_r\right|^p d x \leq c\left(\frac{\mathcal{H}^{n-1}\left(J_u\right)}{r^{n-1}}\right)^{\bar{p}} \int_Q|e(u)|^p d x,
$$
where $Q^{\prime \prime}=\left(-\theta^{\prime \prime} r, \theta^{\prime \prime} r\right)^n$
The constant $c_*$ in 1. depends only on $p, n$ and $\theta^{\prime}$, the one in 2. also on $\rho_1$ and $\theta^{\prime \prime}$.
\end{proposition}

An important result to get strong solution of GE is to approximate $GSBD^p(SBD^p)$ functions by  $W^{1,p}$
From: Approximation of functions with small jump sets and existence of
strong minimizers of Griffith’s energy (more useful results in the paper, some existence of strong minimizers...)

Let $f_0(\xi):=\frac{1}{p}(\mathbb{C} \xi \cdot \xi)^{p / 2}$, for $\xi \in \mathbb{R}_{\mathrm{sym}}^{n \times n}$, with $\mathbb{C}$ obeying (1). Let $Q_r:=(-r, r)^n$ and let $Q:=Q_1=(-1,1)^n$.
\begin{theorem}
There exist a mollifier $\rho \in C_c^{\infty}\left(B(0,1) ; \mathbb{R}_{+}\right)$and $\eta, c$ positive constants such that if $u \in$ GSBD $D^p\left(Q_1\right)$ and $\delta:=\mathcal{H}^{n-1}\left(J_u\right)^{1 / n}$ satisfies $\delta<\eta$, then there exist $R \in(1-\sqrt{\delta}, 1), \tilde{u} \in G S B D^p\left(Q_1\right)$, and $\tilde{\omega} \subset Q_R \subset \subset Q_1$, such that
1. $\tilde{u} \in C^{\infty}\left(Q_{1-\sqrt{\delta}}\right), \tilde{u}=u$ in $Q_1 \backslash Q_R, \mathcal{H}^{n-1}\left(J_u \cap \partial Q_R\right)=\mathcal{H}^{n-1}\left(J_{\tilde{u}} \cap \partial Q_R\right)=0$;
2. $\mathcal{H}^{n-1}\left(J_{\tilde{u}} \backslash J_u\right) \leq c \sqrt{\delta} \mathcal{H}^{n-1}\left(J_u \cap\left(Q_1 \backslash Q_{1-\sqrt{\delta}}\right)\right)$;
3. It holds
$$
\left\|e(\tilde{u})-\rho_\delta * e(u)\right\|_{L^p\left(Q_{1-\sqrt{\delta}}\right)} \leq c \delta^s\|e(u)\|_{L^p(Q)}
$$
and for any open set $\Omega \subset Q$ we have
$$
\int_{\Omega} f_0(e(\tilde{u})) d x \leq \int_{\Omega_\delta} f_0(e(u)) d x+c \delta^s \int_{Q_1} f_0(e(u)) d x,
$$
where $\Omega_\delta:=Q \cap \cup_{x \in \Omega}\left(x+(-3 \delta, 3 \delta)^n\right)$; and $s \in(0,1)$ depends only on $n$ and $p$.
4. $|\tilde{\omega}| \leq c \delta \mathcal{H}^{n-1}\left(J_u \cap Q_R\right)$ and $\int_{Q \backslash \tilde{\omega}}|\tilde{u}-u|^p d x \leq c \delta^p \int_Q|e(u)|^p d x$
5. If $\psi \in \operatorname{Lip}(Q ;[0,1])$, then
$$
\int_Q \psi f_0(e(\tilde{u})) d x \leq \int_Q \psi f_0(e(u)) d x+c \delta^s(1+\operatorname{Lip}(\psi)) \int_Q|e(u)|^p d x
$$
6. If in addition $u \in L^p(Q)$, then for $\Omega \subset Q$,
$$
\|\tilde{u}\|_{L^p(\Omega)} \leq\|u\|_{L^p(\Omega)}+c \delta^{\frac{1}{2 p}}\left(\|u\|_{L^p(Q)}+\|e(u)\|_{L^p(Q)}\right)
$$
The constant $c$ depends only on $n, p$, and $\mathbb{C}$.
\end{theorem}

\section{Fracture Mechanics}

\begin{com} Not ready to do it yet but here are the important papers to look at in the future:
\begin{enumerate}
\item Francfort G.A., Marigo J.-J.: Revisiting brittle fracture as an energy
minimization problem. J. Mech. Phys. Solids 46 (1998), 1319-1342.
\item Bourdin B., Francfort G.A., Marigo J.J.: The variational approach to fracture. J. Elasticity 91 (2008),
\item Chambolle, A., Crismale, V.: Compactness and lower semicontinuity in GSBD.J.Eur.Math.Soc.
23(3), 701–719 (2021)
\end{enumerate}
\end{com}
\section{Important modifications  on Shells}
$$\Omega= D\times [-h,h]$$
First Lemma in balls, no Difference:
\begin{lemma}
    Lemma 3.1 Let $n \in \mathbb{N}$ with $n \geq 2$, and let $p \in(1, \infty)$. There exist $\bar{\delta}, c$, s positive constants, depending only on $n$ and $p$, with the following property. For every $u \in$ $G S B D^p\left(B_1\right)$ with $\delta:=\mathcal{H}^{n-1}\left(J_u\right)^{1 / n} \leq \bar{\delta}$, there exists $\tilde{u} \in G S B D^p\left(B_1\right)$ and $R \in$ $(1-\sqrt{\delta}, 1)$ such that
    \begin{enumerate}
\item $\tilde{u} \in C^{\infty}\left(B_{1-\sqrt{\delta}}\right), \tilde{u}=u$ in $B_1 \backslash B_R$, and $\mathcal{H}^{n-1}\left(J_u \cap \partial B_R\right)=\mathcal{H}^{n-1}\left(J_{\tilde{u}} \cap \partial B_R\right)=0$;
\item $\mathcal{H}^{n-1}\left(J_{\tilde{u}} \backslash J_u\right) \leq c \sqrt{\delta} \mathcal{H}^{n-1}\left(J_u \cap\left(B_1 \backslash B_{1-\sqrt{\delta}}\right)\right.$;
\item it holds
$$
\int_{B_1}|e(\tilde{u})|^p d x \leq\left(1+c \delta^s\right) \int_{B_1}|e(u)|^p d x ;
$$
\item if in addition $u$ is bounded, then one can ensure $\|\tilde{u}\|_{L^{\infty}\left(B_1\right)} \leq\|u\|_{L^{\infty}\left(B_1\right)}$.
\end{enumerate}
\end{lemma}
Next theorem does not look like it need to be changed, but is important to see how the constant behave with $\rho$
\begin{theorem}
    Theorem 3.2 Let $n \in \mathbb{N}$ with $n \geq 2$, and let $p \in(1, \infty)$. Given $\varepsilon>0$ and $\sigma \in(0,1)$, there exist $C=C(n, p, \varepsilon)>0$ and $\tau=\tau(n, p, \varepsilon, \sigma)>0$ with the following property. For every $\rho>0$ and $u \in G S B D^p\left(B_\rho\right)$ with $\mathcal{H}^{n-1}\left(J_u\right) \leq \tau \rho^{n-1}$ there exists $w \in G S B D^p\left(B_\rho\right)$ and a set of finite perimeter $\omega \subset B_\rho$, such that $w=u$ in $B_\rho \backslash \omega, \mathcal{H}^{n-1}\left(\partial^* \omega\right) \leq C \mathcal{H}^{n-1}\left(J_u\right), w \in W^{1, p}\left(B_{(1-\sigma) \rho} ; \mathbb{R}^n\right)$, and
$$
\int_{B_\rho}|e(w)|^p d x \leq(1+\varepsilon) \int_{B_\rho}|e(u)|^p d x, \quad \mathcal{H}^{n-1}\left(J_w\right) \leq \mathcal{H}^{n-1}\left(J_u\right)
$$
Moreover if $u$ is bounded, one can ensure $\|w\|_{L^{\infty}\left(B_\rho\right)} \leq\|u\|_{L^{\infty}\left(B_\rho\right)}$.
\end{theorem}

