\chapter{Introduction}
\label{chap:intro}

The pursuit of understanding and characterizing the behavior of materials under various conditions has long stood at the forefront of both scientific inquiry and engineering innovation. This thesis explores important mathematical tools used in elasticity, a fundamental property of materials that allows them to deform under force and return to their original shape upon the removal of that force. The elasticity theory, initiated by Cauchy in 1825 and subsequently expanded through the contributions of many scholars, serves as the bedrock upon which this work is built. It is within this context that Korn's type inequalities emerge as pivotal analytical tools.

Korn's inequalities, originating from contributions made in 1906 \cite{Korn1, Korn2,KohnThesis}, have played a crucial role in analyzing boundary value problems and the theory of linear elasticity \cite{bib:Horgan}. They are central to establishing the existence of energy minimizers and have been instrumental in both linear and nonlinear shell theories. This thesis explores the intricacies of these inequalities further, demonstrating their relevance to contemporary issues in fracture mechanics, particularly within Griffith's model \cite{FracGrif, FracBook, Frac2, Frac3}, and highlighting the control over the asymptotics of the Korn constant, as a matter of extreme importance  \cite{bib:Gra.Tru.,bib:Gra.Har.2, bib:Gra.Har.3,bib:Harutyunyan.3}.

The first result addresses the buckling phenomena observed in cylindrical shells under axial compression - a subject that has historically presented discrepancies between theoretical predictions and experimental observations. Through rigorous investigation, this study bridges the gap between Koiter's theoretical formula \cite{bib:Koiter,bib:Lorenz,bib:Timoshenko,bib:Tim.Woi.} and experimental data, providing insights into the sensitivity of the shell's buckling load to imperfections and its linkage to the curvature of its cross-sectional curve. This work was previously published in \cite{andre} by the author of the thesis and his advisor Davit Harutyunyan.

Advancing into the realm of weighted Korn and Poincaré inequalities, the second focus of this work introduces new variants tailored for plates. These adaptations are crucial for scenarios necessitating the use of polar coordinates or distinguishing between longitudinal and transverse directions, such as in studies involving junctions of massive bodies and thin rods \cite{wKorn1,wKorn2,wKorn3,surveyQuasilinearSystems}. This development not only facilitates sophisticated localization techniques, but also addresses problems with varied boundary conditions more effectively.

In its concluding part, the thesis breaks new ground with the introduction of a Korn inequality for plates concerning Special Functions of Bounded Deformation ($SBD$). This advancement is vital for analyzing fractures in thin domains and for advancing dimension reduction theories in fracture mechanics, paralleling traditional approaches in elasticity. Although this space is essential for fracture mechanics, their mathematical properties have only been studied very recently \cite{Ambrosio1997, RogerBook, RogerPaper}, and the work on Korn-type inequalities for general domains has only emerged in the past decade \cite{kornBD1, kornBD2, kornBD3}. Drawing from these recent advances, this thesis demonstrates how to manage the Korn constant as the thickness of a plate diminishes, providing significant insights into the behavior of fractures in thin domains.

This thesis is structured to unfold a comprehensive exploration of elasticity, Korn's inequalities, and their profound implications in modern material science and fracture mechanics. We start in Chapter 2 by laying the foundational knowledge of material science, elasticity and fracture Mechanics, setting the stage for the specific investigations that follow. Chapter 3 delves into the phenomena of buckling in thin-walled shells, providing a nuanced understanding of its causes and implications. Chapter 4 expands the discussion to weighted Korn and Poincaré inequalities and their applications, while Chapter 5 explores the role of functions of Bounded Deformation in the context of fracture mechanics.

Together, these chapters weave a narrative that not only advances the mathematical framework necessary for understanding material behavior under various conditions, but also highlights the potential for these theoretical developments to inform practical engineering solutions.

\begin{comment}
\section{Paragraph 1}
This thesis consists in two/three interrelated parts. 
\bl{add concise explanation of each topic}

In this chapter we will introduce the main ideas and concepts that will be used  in each part respectively . We will also describe the organization of the thesis and give a brief overview of the main results and contributions to each topics.

\section{Mathematical theory if Nonlinear Elasticity and 
importance of Korn's inequality}
\section{ Shell theory}
\section{Fracture Mechanics}


\textbf{Introduction from paper}
\end{comment}



