\begin{center}
    \begin{figure}[scale = 0.5]
       
        \centering
                \begin{subfigure}{.45\linewidth}
                \centering
                \begin{tikzpicture}[scale=0.45]
    
    
                    \filldraw[fill=gray!40] (0,2) rectangle (3,5);
                    \filldraw[fill=black] (2,3) rectangle (3,4);
                    \filldraw[fill=black] (0,4) rectangle (1,5);
                    
                    \filldraw[fill=gray!40] (6,2) rectangle (8,5);
                    \filldraw[fill=black] (6,2) rectangle (7,3);
                    
                    \filldraw[fill=gray!40] (6,7) rectangle (8,9);
                    \filldraw[fill=black] (7,7) rectangle (8,8);
                    
                    \filldraw[fill=gray!40] (0,7) rectangle (3,9);
                    \filldraw[fill=black] (1,8) rectangle (2,9);
                    
                    %draw a grid 10 by 10
                    \draw[step=1cm,gray,very thin] (0,0) grid (10,10);
                    %outside boundary darker
                    \draw[step=1cm,black,thin] (0,0) rectangle (10,10);
                    
                     \filldraw[black] (0.5,1.5) circle (0.1cm) node[anchor=north, font=\large] {$D_{i,1}$};
                     \draw[black] (0.5,1.5) -- (5.5,1.5);
                     \draw[black,-{triangle 45}] (5.5,1.5) -- (5.5,8.4); 
                     \filldraw[black] (5.5,8.4) circle (0.1cm) node[above, font=\large] {$D_{k,l}$}; 
    
                \end{tikzpicture}
                    
                \caption{For the case that the target cube $D_{k,l}$ is in a good collumn we can take a simillar path as before.}
                \label{fig:examplea}
                
                \end{subfigure}%
                \hspace{0.5cm}
                \begin{subfigure}{.45\textwidth}
                \centering
                \begin{tikzpicture}[scale=0.45]
    
                    \filldraw[fill=gray!40] (0,2) rectangle (3,5);
                    \filldraw[fill=black] (2,3) rectangle (3,4);
                    \filldraw[fill=black] (0,4) rectangle (1,5);
                    
                    \filldraw[fill=gray!40] (6,2) rectangle (8,5);
                    \filldraw[fill=black] (6,2) rectangle (7,3);
                    
                    \filldraw[fill=gray!40] (6,7) rectangle (8,9);
                    \filldraw[fill=black] (7,7) rectangle (8,8);
                    
                    \filldraw[fill=gray!40] (0,7) rectangle (3,9);
                    \filldraw[fill=black] (1,8) rectangle (2,9);
                    
                    %draw a grid 10 by 10
                    \draw[step=1cm,gray,very thin] (0,0) grid (10,10);
                    %outside boundary darker
                    \draw[step=1cm,black,thin] (0,0) rectangle (10,10);
                    
                    \filldraw[black] (0.5,1.5) circle (0.1cm) node[anchor=north, font=\large] {$D_{i,1}$};
                    \draw[black] (0.5,1.5) -- (4.5,1.5);
                    \draw[black] (4.5,1.5) -- (4.5,6.5); 
                    \draw[black,-{triangle 45}] (4.5,6.5) -- (1.6,6.5);
                    \filldraw[black] (1.5,6.5) circle (0.1cm) node[anchor=north, font=\large] {$D_{k,b_l}$}; 
                    
                    \end{tikzpicture}
                    
                    \caption{When the target cube is not in a good collum but it's in a good row we can take a path like shown in figure.}
                    
                \end{subfigure}
                \label{fig:grid2}
    \end{figure}
    \end{center}
     